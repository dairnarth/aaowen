\documentclass{article}

\title{Biography}
\date{}

\begin{document}

\maketitle
\tableofcontents
\pagebreak

\section{A Short Biography}

A short history of the life of A.A.Owen.\\

A.A.Owen had a Welsh father and a Latvian mother.
They met during World War 2 and lived for a short time back in Wales where Albert was born.
The family moved to Zimbabwe (then Rhodesia) when he was very young.
His father worked in African education and the young Albert travelled the length and breadth of that extraordinary country with his father, visiting mission stations and absorbing Africa.
A.A.Owen became politicized very early in his life and his passionate anti-racist, socialist beliefs took root.

He was a prize-winning student at the Royal Academy of Music and also spent two years learning with that colossus of 20th Century Music – Nadia Boulanger.
His piano teachers were  Harold Craxton and, in Paris, Jacques Fevrier.

He taught piano at the Royal Academy of Music for 15 years and taught music theory and history at London’s famous Working Men’s College, following in the footsteps of John Ruskin, Charles Kingsley, the Rossettis, Vaughan Williams and latterly Jeremy Seabrook, in the College’s role of providing working people with a liberal education.
He rose through the College’s ranks to become its Dean of Studies, working with the then Principal, Lord McIntosh o Haringey, to further develop this extraordinary institution.

A.A.Owen moved back to Wales in 1990.
Here he divides his time between teaching, composing and recording.

He is married to Katherine - a violinist.
He has three children – Hywel, a particle physicist, Cari, a film maker, and Alys a Russian-trained classical ballerina.

In 1985 he wrote the Grand Finale for the Halley’s Comet Royal Gala Concert, held at the Wembley Conference Centre.

In 2002 he was elected an Associate of the Royal Academy of Music for achieving distinction in the music profession.

His music and many recordings have been heard, used in film and TV productions the world over.

The key to understanding Albert Alan Owen and his music can be found in how he describes himself.
“I am a composer and recording artist”.
He creates Sonic Sculptures.
For 20 years all the music he has written has only been heard on CD.
He has rejected the Concert Hall as the means of getting his music to his public.
Although his music cannot be reproduced or interpreted on the concert platform, it can be interpreted by performers.
Uniquely – his recordings are deliberately created and structured so that performers from other artistic disciplines can create a work that can be performed.

\pagebreak
\section{A Musical Journey}

A journey through the musical development of A.
A.Owen.
Albert is one of the most innovative and original composers working today.
For 25 years he has led where others follow.
He was a pioneer in the fields of Electro-Acoustic and New Age Music.
Indeed, New Age Music was not around when his first electro-acoustic composition "Mysteries" was written.
This ground breaking piece of music still sounds as fresh and contemporary today as it did to the audience who first heard itin 1973, at the Mary Abbot Theatre, in London's Notting Hill Gate.
"Mysteries" was released on the Apollo Sound label in 1979 to much critical praise, in both the classical and jazz press.

This was just the start! His next album 'Following The Light', was an even greater artistic, critical and commercial success.
First released on the Apollo Sound label, it was later digitally remastered and released on the Famous label.
As part of The Manhatten Collection, this 1982 recording reached number 8 in the British New Age Chart in 1987.
Now, 13 years on, it has been released again! AND it reached NUMBER ONE in the mp3.com Classical Minimalist chart within a week of being released.

As we start the new millenium, Albert is still "ahead of the game".
His CD "Voyager", was presemted in Cannes at Midem '98, and was released on Vigiesse (Rome) in 1999.
In June 2000, 6 CDs were released by TAP Records on the internet at : www.
mp3.com.
As well as Following The Light, three other works reached the top 15 within a week.

He is now working on his next solo album, his most artistically and technically ambitious to date.
As always, this new work exploits the very latest advances in technology, combining electronic sounds with those of traditional acoustic instruments in a truly unique way.

\subsection{His story}

Albert's background contains a fascinating mix of ingredients which, when fused, combined to create his unique musical personality.

\subsubsection{The early years}

He was born in Bangor, North Wales, in 1948.
His Welsh father was a fine jazz pianist.
His Latvian mother was the sister of Albert Jerums, one of Latvia's most celebrated classical composers.
When his family emigrated to Zimbabwe in the 1950s, he was exposed to, and fell under the spell of African music.
His father worked in African education, and Albert travelled the length and breadth of Zimbabwe with him.
As friends of the extraordinary Haddon family, it was inevitable that the young Albert was brought up to be made aware of the twin evils of colonialism and racism.
He was politicized early, and by the age of 12 was actively engaged in political action, as a junior member of Garfield Todd's Central Africa Party.
Many a Sunday he would attend party meetings in the African Townships of Harare and Highfield.
At the age of 15 he joined with others in founding a non racial music venue in Harare (Salisbury, as was).
Here he played with Township Musicians and discovered the Blues and James Brown.
He was a member of Zimbabwe's premier R\&B band - The Plebs.

While all this was going on, Albert was having classical piano lessons, and occasionally he even did some practise!

\subsubsection{London}

Albert, Alan as he is known to his friends, arrived in London to study classical music in 1967.
He was, inevitably, drawn to London's exciting club scene.
This was the heyday of John Mayall, Eric Clapton, The Who and The Kinks.
Albert was also exposed, for the first time, to 20th Century classical music.
He worked hard and, with the help of that fine teacher Harold Craxton, he became a very good classical pianist.
After competing in The Geneva International Piano Competition in 1968, he knew that he needed to study at a higher level if he was going to be a match fot the very best.
So it was that in 1969 he packed his bags for Paris, and went to study with Jacques Fevrier, the great French pianist and teacher.
At this point fate took a hand.

\subsubsection{Paris}

When he arrived in Paris, Albert asked the legendary teacher Mlle. Nadia Boulanger to accept him as a composition student.
He had been having lessons in London with Patrick Savill, who felt that he had promise, if only he could knuckle down and accept the advice that was on offer.
Nadia Boulanger said yes, and almost overnight Albert realised that composition was his true calling.
Under her inspired tutelage he went from strength to strength.
Instinctively, she seemed to understand the contradictory nature of her new student's personality.
Carefully and patiently she helped Albert to become a skillful enough musician to express his unconventional musical ideas and ideals, with conviction and confidence.
After two intense years, financial difficulties forced Albert to return to London where he spent a happy and successful four years at The Royal Academy of Music, winning The Charles Lucas Medal and The Lady Holland Prize for composition in 1974.
He continued seeing Nadia Boulanger periodically up to her death in 1979, and was happy when, on hearing "Mysteries" for the first time, she said that it was "perfect".

\subsubsection{The Charing Cross road}

One morning in 1973, Albert and a friend decided to go into Selmers on the Charing Cross Road.
On the floor was a second hand Fender Rhodes piano.
Albert fell in love, here was the sound for his music, a music that was both classical and popular, African and European.
A music that drew its inspiration from Debussy and James Brown needed a new type of sound, a new type of enemble.
The Fender Rhodes was the sound that could cement the fractured and contradictory elements of Albert's language together.

And so the A.A.Owen sound started.

Within weeks, money was borrowed, instruments and amplification were bought and Erato was formed.
Mysteries was written for Violin, Double Bass, Pianoforte, Fender Rhodes Electric Piano, ARP Odyssey and Korg 700 and 700s.

Since that day, Albert has not looked back.
The arrival of The Rhodes, signalled his divorce from the classical music world and ushered in the musical style that you now all know by the terms: New Age, Crossover, World Music and so on.

Albert has never been adept at self promotion, but thankfully his music is well represented through his many recordings, and over the years, countless numbers of people, all over the world, have heard and fallen in love with his music.

\subsubsection{Politics}

After he left Zimbabwe, Albert did not give up his political ideals, and pursued these ideals for many years as a tutor, and later the Dean of Studies, at London's Working Men's College.
Although he was for 15 years a professor of piano at The Junior School of The Royal Academy of Music, The WMC was his first love.
Here he started and taught several innovative courses on Jazz and The Blues.
His last act before he left London to live in his beloved Wales was to set up The Rock Music School.

He is a tireless advocate of African American and Black Music, whose virtues he extols with a missionary zeal.
He is also an outspoken critic of the musical opportunists who have stolen the musical clothing of African Music for their own careerist ends.

\subsubsection{Wales}

In 1990, Albert Alan Owen went 'back home' to Wales.
He wanted his children to grow up in a community that valued the individual.
The Welsh, like the African, have survived against the odds.
They have kept their language, culture and values in an increasingly materialistic world.

In what better environment can this most individualistic and original composer continue to work?

In Wales, he teaches and composes, recording his albums at his own state-of-the-art studio.

\end{document}

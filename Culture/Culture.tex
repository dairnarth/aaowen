\documentclass{article}

\title{Culture}
\date{}

\begin{document}

\maketitle
\tableofcontents

\pagebreak
\section{The Working Men's College}

Personal thoughts from an ex-member of this college.
(This article was first printed in Cross-Ties in July 1995)

During 1994 in the London Borough of Camden, heartland of the 'politically correct', a motion was put before the Council of The Working Men's College to change its name.
A referendum was held, the proposal was defeated.
The College, the oldest institution of its kind in Britain, would keep its name - until the next time!

Founded in 1854, by F.D.Maurice, Charles Kingsley and a group of eminent Victorians, collectively known as The Christian Socialists, The Working Men's College was, and remains, one of the most fascinating and radical educational institutions in Britain.
Its aims were then, as they are today, to provide a liberal education for the working man.
(Maurice also founded the Working Women's College, later to become the Francis Martin College; the two institutions merged in 1964.)

From the outset social divisions were minimised within the College environment; tutors and students shared the same facilities and only surnames were used.
Tutors, drawn from the professions, Oxbridge and the civil service, gave their time freely, with no remuneration.
Much the same applies today, only half the tutors are "professional" teachers, but all are professionals.
The majority are now paid, but the idea of the voluntary teacher is far from dead.

The practical idealism which underlies the College ethos has attracted tutors of the highest calibre, including Ruskin, Rossetti, Lowes-Dickinson, E.M.Forster, Vaughan Williams and Caradoc Evans.
In more recent years tutors have included Lionel Colloms, the anti-McCarthyite lawyer; Jeremy Seabrook, the political writer and columnist; Prof. Michael Hancock, the geologist; High Court Judge John Byrt; Spanish Civil War veteran Brian Benjamin; Edward DuCann, past chairman of the Conservative Party; and former Labour Leader of the Greater London Council, Lord Macintosh of Haringay.
(Lord Macintosh has also been a Labour Minister)

The prospectus offers courses right across the academic range: politics, religion, languages (including Welsh, Latin and Greek), art, music, dance, mathematics, information technology, and others.

The facilities are excellent and include a self-service restaurant, bar, large common room, a magnificent library, computer suite, full-sized gymnasium, concert hall, a small cinema and the muniments room.
The last contains rare manuscripts and historical documents (including Lawrence of Arabia's letters) and is used by scholars from all over the world.

The College has an orchestra, cricket, bowling and football teams and a very successful running club - The Mornington Chasers.

Many years ago, the College decided that its independent status had to be preserved.
To achieve this, money was (and still is) invested in the City and property.
The income derived from these investments (worth several million pounds) is given to the College as an annual income by the College Trustees - the Corporation.
This money is completely controlled by the College's governing body - the Council.

The advantages are obvious.
No one from outside can tell the College how to spend its income.
Being comparatively rich and independent allows the College to pursue its collective goals without fear or favour.
Throughout the 1980s it was effectively 'Thatcher-proof'.
While all around, adult education institutions and colleges of further education were under severe pressure, the Working Men's College expanded.
Through astute management and very low administration costs, the effect of 'Black Monday' on its income were minimised.
Currently, there are some 2,500 students and 150 tutors.

While all this is very impressive, the most remarkable aspect of the College, and what places it apart, is its unique governmental structure - a sophisticated constitutional democracy.
The Council, the governing body, is elected.
All tutors and students have an equal vote and representation.
Elections are held every year with half the Council standing for re-election.
No-one can serve on the Council for more than three consecutive two-year periods.
All College Officers are also elected : Principal, Vice-Principal, Dean of Studies and Bursar.

The council employs a full-time staff of only five to execute its policy.
Council therefore has complete control of education policy, strategic planning, finabce (£1 million a year) and management.
All decisions are arrived at through motions put before and debated within the council.
It works and has worked for 150 years.

Consequently, the College cannot stagnate.
Its democratic structure ensures that it has constantly renewed itself, changing its curriculum in response to need and demand.
It gives its members the experience of true collegiality and above all teaches them to respect and work with all manner of people, no matter what their political views, racial background, religion or social class may be.

To be a member of the College is a truly civilised and civilising experience and it stands as a shining example to all institutions.

Albert Alan Owen was a tutor at the Working Mens College for 12 years and and was the Dean of Studies from 1990 to 1991.

\pagebreak
\section{Nadia Boulanger Remembered}
The personal recollections of Albert Alan Owen.

I studied composition in Paris with Mlle. Nadia Boulanger for two years in the late 1960s and early 1970s.
I continued visiting her periodically until shortly before her death in October 1979.
The last time I saw her was the Christmas of 1978, she was very frail of body, but her mind and tongue were as sharp and incisive as ever.

I first heard of Nadia Boulanger through reading Aaron Copland's "On Music", an excellent book.
At the time, I was a typically lazy schoolboy living the colonial life in Rhodesia (Zimbabwe), practising the piano reluctantly, composing poor Debussyesque piano pieces, playing in a Rhythm and Blues Band and writing love songs.
Something in Copland's description of her struck a chord.
In the middle (culturally) of nowhere my destiny was decided.

In 1966, I moved to England to study music, the piano primarily.
I was, to say the least, "off the pace".
I was fortunate in my teachers.
Harold Craxton, the grand old man of English piano teaching, took me in hand, and I improved rapidly.
I was introduced to the "real" world of music.
I studied composition with Patrick Savill, who was, likewise, an excellent teacher and purveyor of knowledge, again I improved at a pace.

London in the mid 1960s was, of course, the mecca for a young R\&B man, and I gorged myself on John Mayall et al.
I also went to concert after concert of contemporary music looking for a similar buzz, but found none.
Nothing clicked! Here I was, an apparently conventional music student, learning piano and composition with fine teachers, but effectively a "square peg in a round hole".
Creatively stranded somewhere between the mighty world of Classical Music and my beloved Rythm and Blues.

On the piano, by 1969, I had progressed to a point where I needed to study with a teacher who could guide me towards the highest level, so I went to Paris to learn with Jacques Fevrier.
As soon as I reached Paris, I took Copland at his word and, withot an introduction, asked Mlle.
Boulanger to accept me as a pupil.
She did!

In, what seemed like, no time at all, she helped me (to clumsily continue the metaphor) square my circle.
She helped me come to terms with and, ultimately, creatively exploit my naturally aquired, though inherently contradictory musical personality.
Only a very great teacher could have done this.
Only a sensitive teacher with a truly open mind could have guided one such as me through these crucial formative years.

What was she like, this musical colossus? I hope that these memories of my time with her will give you, the reader, some idea of her genius and her extraordinary, indeed unique, gifts as a teacher and a person.

\subsection{First lesson}

My first lesson with Mlle. started somewhat awkwardly.
I sat next to her at the piano and when reaching down to pick up my case, money fell out of my pocket and all over the floor.
An already nervous me was now even more unsettled.
I spent the next few moments on my hands and knees under the piano picking up my loose change, much to her amusement! She had a mischievious sense of humour.

From the first moments of this first lesson, I knew that Nadia Boulanger was special.
I was gently, but firmly, interrogated.
It did not take her long to find out that I was both ignorant and naive.
I was also painfully shy and quiet.

When I mentioned that I had been doing Bach Chorales, she was outraged.
I was told, most firmly, that doing Chorales should be the culmination of one's contrapuntal and harmonic studies.
They were not to be done lightly; they were not there to be used as mere educational canon fodder.
Here was my first taste of her moral and musical rigour.

A lesson taught and  a lesson learnt.
As we know, Bach's Chorales are on of the highest point of our art.

When I later told this all to Janet Craxton (Harold's daughter and one of this centuries finest oboeists), she told me that the single greatest musical experience of her life, was hearing Nadia Boulanger playing Bach Chorales on the piano.

And thus, I started my formal training in counterpoint.
What counterpoint it was!
None of the slack and sloppy English approach!

\subsection{Early lessons}

I dislike the term, so loved by our U.K. institutions - "music techniques".
What I studied with Mlle.
Nadia Boulanger was musical technique - (singular)! The superficiality, almost to the level of pastiche, that characterises so much of the training today, was not how Nadia Boulanger approached things.

The contrapuntal task she set was simple (sic).
In species,  one had to write several versions of a line against a Cantus Firmus.
No sequences; no repetition; no consecutives between bars etc.
"Everything is forbidden.
" Only go on to 3 parts when you have mastered 2 parts, and so on.

Imagine, if you will, a particular experience I had with Mlle.
I had managed to complete what I thought to be a perfect piece of work.
No mistakes at all.
Her reaction? "Ah, my dear Alan, but it is not beautiful."

I should point out that when I went to Paris, I was also functionally deaf.
Hefty doses of Hindemith's Elementary Training for Musicians, under the watchful eye of the ever patient Annette Dieudonne, effected a cure.

\subsection{A most amusing lesson for Mlle.}

On one occasion, Mlle. feeling that my musical persona could benefit from having more oomph , asked me, as an excercise, to set a dramatic passage from Shakespeare.
When I had completed this task, she asked me to sing it.
I proceeded, sotto voce.
Impatiently, she stopped me.
I was told to make more noise.
(My efforts, even now, can in no way be described as singing.)
She persuaded me to lose my inhibitions.
"Musicians must sing", no matter how unpleasant the result.
"We must never be embarrased, go on, I won't laugh".

Well I sang out with gusto!
I noticed, out of the corner of my eye, that Mlle. was chuckling.
My vocal efforts were that awful.

Strange thing, since that day I have never been inhibited in the slightest by my terrible voice.

\subsection{A painful lesson for me}

My lesson on one occasion took place immediately before her famous Wednesday afternoon class.
Towards the end of my time, as all were gathering in her anti-room (full of wonderful photographs given to her by many of the 20th Centuries greatest figures), Mlle. asked me to play an harmonic sequence and run it chromatically through the keys.
I couldn't do it!
Again and again I came unstuck.
Again and again she asked me to try again.
"How can you call yourself a musician, if you cannot perform such a simple task."
No good, I still couldn't do it.
Half an hour into the time when the class should have begun, she gave up on me.

I felt dreadful, all her other students had heard me, and now I had to sit with them for the next couple of hours, a musical failure!

Cruel?
Yes!
Was she justified?
Again, yes!
Music demands certain skills.
If you don't have them, there is "no hiding place."

This is just one example of how tough she could be.

\subsection{Long hair, kites, hands, servants and short sight}

On one occasion, Mlle. noticed that my hair was getting rather long.
It was very fashionable in those days, and I was, as my own children are today, a "fashion victim".
I was also wearing a rather lurid pair of purple trousers.
"Alan", she admonished, "one must distinguish oneself through individuality and not singularity."
She was always watchful of her pupil's personal development, nothing went unnoticed.
Like all great teachers, she knew that it was not enough just to teach the subject, one had also to teach the person!

A most notable characteristic of Nadia Boulanger's teaching were the fascinating and unpredictable tangents that the lessons often took.

She once confessed to me her fantasy to go and fly a kite.
Like a young child, her face lit up as she speculated on the joy that she would feel should the chance present itself.

On another occasion, we took time off to wonder at the construction of our hands.
Our bodies were a God given miracle.

She described what it was like to be short sighted.
Mlle., when I knew her, was very short sighted and used a magnifying glass with an in-built light to study scores.
Rather than bemoan her handicap, she perceived the distortions that she saw as an abstract source of mysterious and beautiful images.

Typical of her time and social position, Nadia Boulanger had servants; a delightful family who shared her apartment and looked after her and us students.
During the course of a lesson, she asked me to lower a blind."
I had better not ring the bell, as I think he is rather annoyed with me."
It had obviously been a hard morning and Nadia was showing a sensitivity to another person quite untypical, in my experience, of the French upper class.

\subsection{Nadia Boulanger on other musicians}

Mlle. was quite forthright in her opinions of other musicians.
When giving a master class at The Royal Academy of Music, she was scathing in her criticism of all but one (A piece by Morris Pert (who went on to become one of the world's leading percussionists)),  of the compositions that were presented to her.
Attempting to shield one of his students from her stinging remarks, a particularly distinguished Academy Professor was put in his place in no uncertain terms.

Befitting a musician of her quality and stature, her judgements carry an enormous authority.

For the record:\\

\begin{tabular}{p{0.25\linewidth}cp{0.6\linewidth}}
    Stravinsky                         & -- & The greatest 20th Century composer.\\
    Menuhin                            & -- & Gave Mlle. her most memorable and moving experience when she heard him as a young boy.\\
    Boulez                             & -- & The finest ear of any living musician.\\
    Dinu Lippati                       & -- & On the anniversary of his untimely death, Nadia devoted her class to him and his work. His integrity was held up as a beacon to us students.\\
    Messiaen                           & -- & On his teaching methods, she charged him with producing "lots of little Messiaens."\\
    Barenboim                          & -- & Mlle. had a very high opinion of his work with The Paris Orchestra, which she felt had been transformed by his musicianship and rigour.\\
    Jacques Fevrier (my piano teacher) & -- & On hearing that I was a pupil of his, she said "It is best that I say nothing!" He, in turn told me to ignore anything she told me about piano playing - "The woman is mad!" Some history here!\\
    Zimmerman                          & -- & A great pianist.\\
    Ashkenazy                          & -- & Not a great pianist.\\
\end{tabular}

\subsection{Nadia the musician}

Her phenominal skills are well known and documented.
I was and remain in awe.
She told me that she had been able to play The 48 (Bach Preludes and Fugues), by memory ,at the age of seven.
I suggested that for me this would be impossible!
Nonsense she replied.
"Do one a week, it will take you less than a year."

My recollection, is that Mlle.
knew everything - every significant piece by every significant composer - by memory! Imagine that! No musical task was beyond her.

She once told me why she did not compose.
I paraphrase - "Although I could write better music than most of what is being written today, I do not, because I am not a real composer."

This is an attitude that I endorse wholeheartedly.
Merely being able to write music effectively, does not mean that the writer is a composer.
Who is and who isn't is, like Pirsig's "Quality" (Zen and The Art of Motorcycle Maintenance) is easy to recognise but very difficult to define.

\subsubsection{Valery and Stravinsky}

It seems to me, that these two men were amongst the most significant influences on Mlle.

I would go so far as to say that in the case of Stravinsky, Nadia's devotion clouded her, in other ways almost infallibe, sense of judgement.
He could do no wrong!
Certainly, Les Noces is one of, if not The, greatest works of this century, but surely the turgid Symphony in C is one of the worst to be written by a major composer.

Valery is too much for me, I am simply not clever enough to cope with his work.

\subsection{Nadia and my music}

It will be obvious by now, that I was a difficult pupil for Mlle.
When I came to her, I was not a proficient musician, let alone a skilled one.
Also, I was instinctively "out of step" as far as compositional orthodoxy was concerned.

"The problem with you music Alan is too much parallelism".

This "problem" is the basis of my style.

Thus, she was faced with the task of helping me to become a skilled composer, whose core values were in direct conflict with what she knew, valued and understood.
Over time, she came to appreciate the problem that I faced.

"God help you, when you understand what you are doing!" she said in one of my lessons.
She understood that I was on my own, technically, musically and, to some extent, culturally isolated, with no technical or historic foundations on which to fall back.
It is testimony to her teaching, that the note manipulating technique that she, literally, drilled into me has enabled me to move myself forward.
Her example, as an exemplar of the values of integrity, rigour and moral courage that were a byword for her life and teaching, is always with me.

\pagebreak
\section{Wales from Within}

An appreciation of the people, landscape and culture of Wales.

\subsection{A Continent in a Country}

Wales is tiny, but it is one of the most fascinating places on Earth!
Its history is a near miraculous story of survival against the odds.
Even though its most remote corners are less than 120 miles away from its mighty neighbour - England, Wales has managed to retain an unique cultural and linguistic identity.

\subsubsection{Why `A Continent in a Country'?}

Firstly, The Landscape.
When you travel the length of Wales, the thing that most strikes the visitor, is the quite astonishing variety of its landscape.
From the, now sad, valleys of the south through the rolling hills of Mid-Wales to the magestic mountains of the north, no two scenes are the same.
Every view, every valley, every town and village, has its own unique character and atmosphere.
All are clearly Welsh ,yet all are different.
From the modern city of Cardiff, to the elegance of Aberystwyth, the bustle of Carmarthen, the charm of Llanidloes, the historic town of Harlech and the grandiose pomp and splendour of Llandudno; where else could you find such contrasting places in so few square miles?

Then there are its people.
Wales is populated by fiercely proud small communities, each with its own history, its own particular story.
The once thriving mining areas of the industrial English speaking south, where the spirit remains strong and optimistic.
The farming heartland of Carmarthenshire and Ceredigion, inhabited by Welsh speaking farming communities who in many cases go back generation after generation.
Then on to the north, a place of mystery, a harsh and often hostile environment whose inhabitants have long carved a difficult living, one way or another, out of the hills.

All these people, no matter what their particular experiences and heritage, are all bound together by their Welshness, and a sense of common identity.

Unusually, this sense of nation seldom manifests itself in a narrow nationalistic way.
The Welsh have always been extordinarily tolerant and welcoming to the many English people who have come to live amongst them.
The Welsh have traditionally embraced the world and all its challenges.
Indeed their contribution to British History is incalculable, particularly in the fields of politics, religion and the arts.

Now, Wales is taking a leap forward towards a political nationhood, which will, I am sure, become another inspirational chapter in its long and varied history.

\subsubsection{What is Wales and why is it special?}
As you travel from England into Wales, there comes a particular point in the journey where you know that you are in a different place.
Whether one arrives in the north, south or middle, this is always the case.
This point is somewhere near the actual geographic boundary but not always precisely at it.
All of a sudden, the whole feel of the surrounding countryside changes.
Romance and mystery take over.
Un-pronouncable, to the stranger,and evocative road signs inform the traveller that he has arrived in a living enigma.
How is it possible that England and Englishness stops so abruptly?
Somehow, as if by magic, the traveller is subconsciously ensnared by Wales.
Many visitors never leave, almost all return, again and again.

Yet as soon as one tries to define Wales and The Welsh, it is like trying to keep hold of a bar of wet soap.
The qualities that add up to Wales are illusive.
No two definitions are exactly the same.
In my experience, like nowhere else on earth, the Welsh are continually exercised as to their identity.
Each Welshman and Welsh Woman defines himself and herself individually, and it is this individuality that lies at the heart of  being Welsh.

Wales, the land of the individual!

\subsubsection{Non-conformist tradition}
Non-conformity lies at the root of Welsh life and history.
Roman Catholicism underpins most European History.
Even in Protestant England, if you scratch the surface the Catholic root is laid bare.
In Wales the story is quite different.
Scratch its surface and you find the old Celtic Church, Catholicism is nowhere to be found.
Even in St. Davids, the feeling is not the same as it is in other historic religious sites in England and Europe.
This absence of an obvious Catholic history makes Wales unique amongst the Celtic nations.

Wales was never 'priest -ridden' in the way that Ireland was/is, and yet, religion is part of the very fabric of Welsh History.

Travel through Wales and you see Chapels: each village, each hamlet, each settlement boasting  one or more.
Read the names.
So many versions of Protestantism: Methodists, Presbeterians, Baptists, etc. etc., almost as if each village had its own denomination!
Even now in today's secular world, religion and religious observance is a very important, fundamental, part of Welsh life.

It is this variety of religious experience that underlies this nation of individuals.
Each and every Welsh person has a sense of of their own value and worth.
The 'class system' never took root here in Wales.

Traditionally people in Wales are valued for their personal qualities and not for their material status.
This still, in my experience, holds true today.
Wales can, therefore, be defined as a country which has better values than other places.
It is these shared values  that define The Welsh.

\subsubsection{The Eisteddfod: the glory of Wales}
The Eisteddfodau (plural) are the glory of Wales.
They offer an almost unique (in this day and age) opportunity for artistic activity and participation.
All the people of Wales are able to take part in the whole range of artistic disciplines: literature, drama, music and dance.
From the smallest junior school and smallest community to the impressive national gatherings, the Eisteddfod is, for many people, the highlight of their year.

Where in other societies the youth are inclined toward the cynical and philistine, here in Wales one sees young people taking part, without inhibition, in poetry recitations, literary contests, plays, concerts and the like.

It is at The Eisteddfod that The Welsh Language and Culture are celebrated, honoured and renewed.
A disparate nation coming together, on a national level; a community coming together at the level of the village.

It is a genuinely "grass roots" affair.
Indeed during the national Eisteddfod, which is located in a different region every year (a kind of travelling road show), televisions the length and breadth of the country are continually tuned in.a

\end{document}

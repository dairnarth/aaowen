\documentclass{article}

\usepackage{makecell}

\title{Contemporary Music}
\date{}

\begin{document}

\maketitle
\tableofcontents

\pagebreak
\section{Cultural Piracy}

Here, I will tackle one of the thorniest topics concerning contemporary music: culture and race.
I know that the predominant musical cultural force in this century is Black American.
There is no doubt about this!
Yet this fact is almost never acknowledged amongst the white dominated cultural elite.

Although the sources were rarely acknowledged, almost from ``day one'', Ragtime, Jazz and the Blues were appropriated by white composers and musicians.
These composers and musicias laid claim to these and subsequent (African American) genres as part of their own heritage.
A culture rooted in suffering, born of the unquenchable spirit of black people was hijacked.

When this was done out of wonderment, as was the case with Debussy and Ravel, all well and good.
(After all, these two great composers, through their influence on the great songwriters of the 30s, 40s, and 50s, amply repaid their debt via the ``standards'' that became a staple for Jazz musicians.)

When this is done ``on the street'' by the youth of today, fair enough.
``On the street'', in the big cities, many young people have, long since, become colour blind.

When it was, and is, done to bolster and pep-up an artistically and commercially failing culture, without any heed being given to its different underlying cultural values, this represents a cynical type of cultural colonialism.
It was, and is, immoral!
It is somewhat akin to theft.

When children are taught that perennial favourite ``The Golliwog's Cakewalk'' by Debussy, are they educated about the culture that gave birth to this style; a culture who's musical and compositional ideals so captivated the imagination of Debussy and Ravel ,that their music was forever changed by their exposure to it?
\textbf{No they are not!}

When we rhapsodise (sic) about Gershwin's genius and Bernstein's brilliance, are we ever made aware of the fundamental debt they owe to black culture?
\textbf{No!}

Does the ``Rock and Pop'' generation in the white suburbs have any idea where this music all started?
\textbf{Again no!}

Real art, comes from one's own experience, or from some kind of profound spiritual empathy with someone else's experience, an empathy so strong that it fundamentally changes one and the way one thinks!
It is not enough to simply: wear the clothes, affect the style, walk the walk and talk the talk.

The real artist lives the real life.
To play like John Coltrane, one must BE LIKE John Coltrane.
Some white jazz musicians, a very few, have achieved the life and therefore the right to play the music, but only a few: Bill Evans and Art Pepper are among this small group.

Some ``serious'' composers (Terry Riley for example) have embraced the ideals and philosophy of African/American musical thought, and left the values of the classical world, with all the associated career prospects, behind, but not many.
Most who draw from this well, are mere opportunists, ersatz dabblers, with an eye to the main chance: the type of people who need the comfort of the ``safety net'' that academic based European culture provides but who enjoy the odd vicarious thrill.
They never, metaphorically speaking, stray far from home.
At best, their's is a sentimental attachment, a middle class, self deluding romance.

\textbf{Be warned!}

What you are witnessing in today's classical music scene is nothing less than the most blatant type of cultural fraud: an easily attained, quickly applied veneer, with the affectation of comprehension.
It is acquired after taking a superficial, educational ``package tour'' of various stylistic locations: If its Monday it must be African Drumming, Tuesday - it is Throat Singing and Wednesday is Gamelan Time.
Thus, by taking a few short steps, an essentially uncomprehending mind clothes itself in another tradition's vernacular.
The results are crude pastiche.

Naturally, financial reward, and artistic credibility goes to these ``Cultural Magpies''.
The originators, almost always, remain largely unknown and unrewarded.

Who remembers Arthur ``Big Boy'' Cruddup, the man who recorded ``That's Alright Mama'' in 1947!?

Nothing changes!

So called World Music is no more than another scam; designed to shift product and confer ``coolness'' onto both its audience and its practitioners.
They would switch to Martian music, if there was money to be made, a trend to be followed, or a career to be furthered.

Its all showbiz, it is all phoney ``flavour of the month stuff''.\footnotemark

    \footnotetext{This month's flavour is: wishy washy religious music with nice chords and meaningful titles.
    Deeply spiritual (of course), and usually accompanied by a video (soft focus, naturally)}

\subsection{How to spot the phoneys}

This is easy, they dilute the form to taste.
They have no stomach for the real thing, and nor do their audiences!
Their's is a processed, pureed music, easily digested and easily forgotten.
Good music is hard!
Hard to play and hard to listen to.

It takes many years to become a proficient classical musician, but apparently only one term/semester to master African Music!

Remember: Nobody ever went broke overestimating the public's taste.

European culture has had its day, but white political and cultural power is still in place, supported by its educational infrastructure and its subsidised ``Arts Professionals''.

So we'll all have to put up with cultural racism and exploitation for a little while yet!

\textbf{However, dig deep enough, and you will still be able to find ``the real thing'', played by real people!}

\pagebreak
\section{20\textsuperscript{th} Century Music: 1}

An alternative view of 20\textsuperscript{th} century music.
From as early as I can recall, 20\textsuperscript{th} Century Classical Music has been in a state of crisis.

Composers bemoaning the lack of audiences.
Audiences bemoaning the lack of good tuneful music.
Musicians complaining that there is little new repertoire to stand alongside the established masterpieces of history.
Promoters complaining about falling or low attendances.

Since my student days it has ever been thus.
I have attended, and taken part in, innumerable meetings where contemporary music professionals have looked for ever more "imaginative" schemes to help promote New Music- usually theirs, all to no, or very little, avail.
Performers "dressing down"; pre-concert talks; educational initiatives and, of course, large amounts of public and private subsidy have all failed to rouse the public at large.

Only the already converted attend New Music concerts regularly and with any degree of enthusiasm.
The majority of music lovers remain stubbornly indifferent to New Music.

It is my view that these problems may never be resolved.
It is New Music itself that is the problem.
No amount of education, subsidy or innovative programming will ever provide New Music with a secure, self supporting and artistically viable audience base!

I should say, at this point, that I am no musical Luddite, continually looking back to a "Golden Age" of tuneful harmonious security.
Therein lies the road to vacuous banality.
Indeed, I have very little sympathy with hose who publicly, and sometimes virulently, attack contemporary composers.
Preaching from their own parochial soap boxes, their arguments, such as they are, are usually informed by ignorance and bigotry.

It is a fact that much of what has been composed in the last half-century has been artistically and creatively outstanding.

The problem, is that hardly anyone is listening!

\subsection{An analysis}

The first half of the 20\textsuperscript{th} Century was a time of great promise, excitement and extraordinary achievement.
The music of Debussy, Stravinsky, Schoenberg, Ives etc. etc., pointed out several ways forward which would continue the great European/Western musical tradition.
The later music of Varese, Webern, Messiaen etc., likewise, bristled with innovative techniques and technical rigour.
Optimism and confidence were the order of the day.
Although the music was often "difficult", the future seemed assured.
In time the wider public would come to love and appreciate this modernity in all its glorious variety.

Unfortunately this did not, and has not happened.

Early enthusiasm for, and confidence in, New Music has gradually been dissipated and eroded.

Why?

One of the principal causes, lies in the linguistic diversity that is a hallmark of Twentieth Century Music.
The fact that there is no common language is at the root of the problem!
There are too many linguistic codes for even the most informed and sympathetic listener to crack!

What can explain this phenomenon, where New Music is so often rejected, "returned to sender", neither loved, understood or appreciated?
How did this situation come about?
Is there an underlying reason, a flaw in the logic that this centuries’ composers have been following?
A flaw that lies at the very core of New Music’s problems?

I believe that the answer to these questions is emphatically yes!

Until and unless this situation is honestly addressed and recognised, the contemporary composer’s twin blights of; injured introspection on the one hand, and intellectual arrogance on the other, will continue, unresolved.

LOGIC AND WHERE IT GETS YOU!

The assumption that the work of the early 20\textsuperscript{th} Century masters was the next step forward in music’s progress was logical and rational.
As night followed day, Modernism would replace Romanticism, just as Romanticism had replaced Classicism.
This inexorable logic, so brilliantly explained by writer and critic Andre Hodeir*, contains one fatal flaw: the myopic belief that European/Western Art Music has a guaranteed future in a multi-cultural, modern world.

"It ain’t necessarily so."

It certainly is not!
No matter what composers and their apologists do and say, the public (even the European public) has increasingly (as this century has progressed) shifted its allegiance to music with a different cultural root.
Very little 20\textsuperscript{th} Century Music has entered the standard repertory.
Only the, I believe temporary, phenomenon that I call "New Musac" can be said to buck this trend.

Outside the narrow introspective confines of  the salon and "Academe", New Music is, quite simply, a non-starter, irrelevant!

*Since Debussy - A View of Contemporary Music by Andre Hodeir

published by Secker \& Warburg 1961

\subsection{Some observations}

The World is no longer subject to an exclusively European artistic hegemony.
The world has moved on!
We are, all of us, on a different journey now!

Contemporary composers have simply been overtaken by events.
We now live in "The Global Village", with all its associated implications.

Composers must communicate in the appropriate language if they want to make contact.

We may mourn the passing of time and the loss of old cultural certainties, be they European or Nationalistic, but, in seeking to maintain, promote and develop an European culture (albeit with a leavening of other traditions and processes), we are living in a false world of delusion, bathed in the warm comforting light of "cultural afterglow".

Other art forms and creative disciplines have recognised this, and prosper.

If all of the above is true, ( and I, obviously, am convinced that what I have written is self-evident and that my analysis is correct) why then has this situation not been addressed before?

The answer lies in vested interests!

\subsection{Vested interests}

It is not in the interests of "The Powers That Be" to acknowledge these changed circumstances!
They have a vested interest in maintaining the status quo.

Imagine a situation where New Music was both artistically and commercially (sic!) viable:

What would happen to all of the committees?

What would become of Arts Councils and their staff?

What would happen to all those people who have power in the world of contemporary music?

Imagine a situation where academic institutions no longer held the exclusive rights to our cultural definitions:

What would become of teachers who were only teaching the irrelevant and the esoteric?

What would become of their powers of patronage?

What would they do for a living?

RIGHT!
IT IS UNIMAGINABLE!

This culture of a mutually rewarding interdependence precludes the possibility for change.

Therefore, our young composers grow up and train in a protective, but ultimately stifling and creatively proscribed, environment.
Each incarnation, increasingly, a paler clone of a previous generation.

Welcome to the new orthodoxy, the ultimate madness: where success is measured, not by how little subsidy one needs, but, by how much one gets!

CRAZY !

SOLUTIONS

BUT FIRST:

\subsubsection{Albert’s little history lesson (1)}

Once upon a time, around 1900 in fact, functional tonality died!

The stresses and strains of its wonderful contradictions had finally torn it apart.
The collective, wondrously sustained, creative assaults by Bach, Mozart, Beethoven, Chopin, Liszt, Wagner and the younger Schoenberg left it exhausted of possibilities.
There was nowhere left to go!

The 20\textsuperscript{th} Century masters tried various possible ways forward.
Much great music was written.
Each had his disciples.
Each had many virtues.

Each was a "Dead End"!

\subsubsection{Albert’s little history lesson (2)}

In The West, around 1900 in fact, a new musical language emerged in unison with a new geo-political reality.
It resulted from a perfectly natural cultural synthesis, rooted in history, inevitable.

This music was called The Blues.

The new power?
The United States of America.

\subsection{The blues}

The emergence, history and background to The Blues is well documented.

That this music forms the root of all the contemporary "popular" styles is likewise, commonly acknowledged and universally recognised.

Why then is it not studied in any serious way in our (U.K.) Academic Institutions?

After all, Early Music, Indian Music, Gamelan, African Drumming, Jazz, Pop, Rock, etc., etc., you name it, are all assiduously, albeit usually on a superficial level, studied.

Could it be that, because the music that gave birth to our popular styles is not European/Western is considered degenerate and is, therefore, not deemed to be a suitable subject for those with artistic sensibilities and serious intentions?

Whether this is evidence of cultural racism, or just ignorance, I do not know.

\subsection{DEBUSSY AND RAVEL}

Both these French masters understood the significance of The Blues, and their music, particularly that of Ravel, reflects this fact.
Their individual and separate influence on 20\textsuperscript{th} Century music is usually only understood from a purely European perspective.

The rest of this article (polemic) is concerned with these composers and their unsung contribution to the only viable musical language that we have for this and the 21\textsuperscript{st} Century.
A language that, if properly understood, could support and sustain serious art music as well as the more popular forms that are, currently, its predominant legacy.

That their contribution was not a conscious one, in no way invalidates the following thesis.

To follow: A technical/historical analysis of The Blues, and the "Ravel effect".

\pagebreak
\section{20\textsuperscript{th} Century Music: 2}

Linguistic attributes of functional tonality versus African music.
In order to understand my claims that African/American Music is the natural language for the 20\textsuperscript{th} and 21\textsuperscript{st} Centuries, and that it demonstrably has the potential within its vocabulary/grammar for serious composition, I must first give an analysis of its precursors Functional Tonality and The Blues.

The values that underlie these two languages are diametrically opposed.

BUT FIRST:

\subsection{Albert’s little history lesson (3)}

Because of The Greeks, Western Music took a different path from the music of other cultures.

The mathematically based system of pitch which was developed/invented by The Greeks meant that for the next 2000 years there would be two kinds of music: natural and unnatural/artificial.

This value (artificiality) and the values thus derived, governed the course that Western Music took from Greek times to the start of the 20\textsuperscript{th} Century and beyond.

\subsection{Values of Western music versus non-Western music}

\begin{tabular}{p{0.45\linewidth}|p{0.45\linewidth}}
    \textbf{Western music} & \textbf{Non-Western music}\\
    \hline\\
    Artificial, idealised Sound; & Natural Sound;\\
    Need for originality and innovation in the light of scientific and social advances; & Natural Evolution;\\
    Composed music of finite length; & Not composed;\\
    A written/taught tradition; & An aural tradition;\\
    In time this came to mean as a consequence of the leading note:\footnotemark Contrary motion; & Parallelism;\\
    Structural interdependence between melody, harmony and rhythm; & Independence of melody, harmony and rhythm;\\
    Rhythm implicit; & Rhythm explicit;\\
    Music as an abstract form of expression -- socially elitist; & Music as an integral part of social life -- communal;\\
    Standardisation of pitch, fixed; & Random pitch, not fixed;\\
    Invention of instruments (technologically driven input); & Traditional instruments;\\
    Musicians seen to be innovators and intellectual/philosophical pacemakers; & Musicians seen as either spiritual leaders or "outside the pale" - depending on individual cultural traditions;\\
    Progression in the time/space continuum; & Static, non progressive;\\
    Tension achieved through thematic/rhythmic development; & Tension achieved through accumulation and repetition;\\
    Dynamic forms; & Passive forms;\\\\

    \multicolumn{2}{c}{Eventually:}\\\\

    Interpretative performance tradition. & Non-interpretative.\\
\end{tabular}\\

    \footnotetext{As a consequence of the leading note.}

\subsection{Functional tonalty}

In the interests of brevity, I will not attempt to trace the evolution of Functional Tonality from The Greeks to the time of Bach.
There are many excellent books and articles on the subject, written by people who are far more learned than I.
My purpose here is to analyse it, and show its effect on European Music from Bach's time onward.

I will only say that during the period of time between of The Ancient Greeks and Bach, the European Ear became gradually conditioned into accepting Function Tonality as "natural".

\subsubsection{The concept of the leading note}

The concept, idea, of the leading note lies at the root of Functional Tonality.
Indeed it is the root of Functional Tonality.
Everything springs from its effect.

Put simply, the fact that the leading note draws the ear inevitably towards the tonic, means that a sense of gravity, home, is created.
This in turn leads to motion, thus generating regular rhythms, and a sense of direction when the pitches are drawn from a particular scale (the major in particular).In turn these pitches beget harmony and harmonic progression.
Inevitably this harmonic progression is obliged to be synchronious with a regular, arithmetically simple,rhythmic pulse.

Out of sync = syncopation = instability = Gravity Defied!

\subsubsection{The major scale}

The major scale is the stable scale.
Its internal organisation is consistent, regular.
Its primary triads are all major - all the same.
The primary triads 1V and V becoming 1 of the next key along, in the circle of 5\textsuperscript{th}s.

\subsubsection{The minor scale}

The minor scale is inherently less stable.Its internal organisation is more confused.
Its primary triad V is not the tonic triad of its dominant minor.
This leads to feelings of insecurity and therefore tension.

\subsubsection{Major versus minor}

The consequences of having these two structures living side by side, are felt by all of us who are conditioned to listen to Classical Music.
Secure Major allows us to be happy, positive, optimistic, relaxed.
Insecure Minor makes us tense, sad, worried.

The up-front world of the Majors' Circle of 5\textsuperscript{th}s, is shadowed a minor third lower by the emotionally anxious Circle of the Minors.

This arrangement of functionally organised pitch is possibly mankind's greatest achievement.
A structure that mirrors our perfectly regular, yet somewhow puzzlingly irregular, Universe.
Matter and Anti-Matter: Each note orbiting its own tonic, within the larger orbit of the key system; two scales types moving in parallel.
A glorious abstraction yet absolutely human.
No wonder it was able to provide the means for, almost certainly, the greatest body of creative work that mankind has ever achieved: Classical Music.

Within this system, however, lay the seed for its destruction - Chromaticism.

In the space of 250 or so years, chromaticism had destroyed the system of Functional Tonality.
Gravity was defied, and with it the certainty and security of the old order was no more.

The very system that enabled individual composers to express their individualism so precisely, that encouraged so many uniquely personal utterances, was destroyed by the very individuality it allowed and encouraged.

How quickly it all happened.
Classical Music was one glorious adventure after another:

The work of J.S.Bach, in which he exploited every loophole, examined every possibility, testing the system to the very limits that his contrapuntal style would allow.
The brilliant naughtiness of Scarlatti and the formal inventiveness of Handel, soon gave way to the elegant, melodic, classicism of the Mozart, who did all in his power to detatch his sinuous, devious melodies from their harmonic moorings.
Beethoven, in a creative fervour that lasted only 30 years, overwhelmed classicism in his desperate need to have his music encompass all that life meant.
The  traditional classical forms, the sonata and the symphony, could hardly contain the visions of Brahms and Schubert.
So when Chopin and then, that true revolutionary, Liszt got to work, the end was in sight.
It took a few short years for the other great Romantics like Wagner and Tchaikowsky to finish the job.
A few "tail enders" found something new to say, but essentially Functional Tonality was dead.
R.I.P.

\pagebreak
\section{20th Century Music: 3}

If Functional Tonality represents mankind's greatest artistic achievement, the Blues can be seen as one of the most most significant consequences of mankind's development.
In the Blues we see a miraculous coming together of two apparently incompatible and technically irreconcilable languages and traditions: those of Western Art Music and African Music.
The post Greek world of Western artifice merged with the naturalistic world of Africa to bring us a musical language which is now, in its many manifestations, the completely dominant language in today's musical world

What and how this happened is something we can never know with exact historical precision!
The particular timetable of the wheres and the whens must by deinition remain speculative.
All we know for certain, is that it did happen, and that it happened before the turn of the century.

Why this coming together happened though, is easy to understand.
Indeed, with the benefit of hindsight, it was as inevitable as 1 + 1 = 2!

HOW THE BLUES CAME TO BE

(OUT OF THE "MELTING POT")

When the Europeans colonised North America they took with them two significant pieces of cultural luggage: Functional Tonality and Christianity.

Functional Tonality when combined with Christianity came in the form of Hymns.
These Hymns though simpler and  less chromatically complex than the Chorales of Bach, were firmly rooted in the typical triadic forms of 18\textsuperscript{th} Century tonality.

The need for massive amounts of labour to service the economies and sustain the growth of the new colonies, gave rise to the Slave Trade.
These slaves were drawn, predominantly, from West Africa.

These men and women from Africa, in their turn, brought their own musical forms and cultural luggage.

It is worth noting that culturally and musically, much of this part of Africa was itself an amalgam of two traditions: Native African and Islamic.
The musical language which the slaves brought with them was, therefore, also rather more complex in form than one would at first think.
Ornate melodic and rhythmic processes were typical of an extremely rich and varied tradition.

For their own good (sic) the slaves were converted to Christianity and were exposed, in the form of hymns, to Functional Tonality "in the raw".
Thus, along with their conversion to Christianity, they were also converted to the three primary triads.

The notion of "harmonising", or singing simultaneous parallel lines, was not an alien concept to these uprooted people.
(It is worth noting that, even today, in Africa,traditional choirs sing in a recognisably harmonic way, without necessarily following the "rules" of harmony.)
The Africans, therefore, took the three chords which form the basis of Western Music easily into their own music.

Assimilating the major/minor key system from the melodic stand point, with its leading note and inbuilt heirarchical structure, was quite another matter!

The notion of the leading note and its directly felt consequences (that is: by Europeans), was completely alien to the African musical psyche.
Their pitch systems were not "fixed".
Indeed having a flexible sense of pitch is, in some ways,  the "whole point" in African Music.
Sharpness and flatness are part of the exressive nature of this language.
In Functional Tonality being "out of tune", is seen as bad, wrong and a sign of incompetence - something you do not do!
The relationship between scale and harmony is likewise fixed.
In African music there is no fixed/technical relationship between melody and harmony.
Therefore,unable to assimilate the Western scale and its leading note, the slaves' exposure to the major/minor scale resulted in an amalgam of their own, natural, pentatonic scale and the western scale system.

Recipe: major/minor mixed with pentatonic = The Blues Scale.

All of the above resulted in a  melodic scale that could be accompanied by chords that contained different notes, and because these chords were tonal the bar line with its regular, arithmetically simple pulse made an appearance.

Eureka: A Miracle.
West + Africa = The Blues.
At a stroke (over a couple of hundred years or so), the Western, African and Islamic Traditions were welded together, later to be joined by Euro/Islamic(Arabic) Spanish music via Mexico and The Carribean (but that's another story.)

"NOW THAT'S WHAT I CALL WORLD MUSIC!"

TECHNICAL BIT: A looks at the notes.

Note: The typical text book pentatonic scale is a quite arbitrary choice of notes, based on the major or minor scales.
5 into12 doesn't go!

My Pentatonic is: (from C) C,Eb,F,G,Bb,C.
Insert D and A, and you end up with the Blues Scale.

Thus:

\begin{tabular}{ccccc}
    \makecell{Major\\(C,D,E,F,G,A,B,C)} & + & \makecell{Pentatonic\\(C,Eb,F,G,Bb,C)} & = & \makecell{Blues\\(C,D,Eb,F,G,A,Bb,C)}
\end{tabular}

\subsection{Two observations}

\begin{tabular}{lcp{0.8\linewidth}}
    Q. & -- & Why are the Blues so called, and why are they, wrongly, perceived as sad?\\
    A. & -- & The minor melodic intervals which are a part of the Blues scale are to Western Ears associated with sadness. This, plus the sentimentalized, collective sense of guilt which pricked the conscience of an ashamed white people, caused Europeans to project their western sensibilities onto this music. In so doing, they missed the point!\\\\
\end{tabular}

If God was white, The Blues really was "The Devil's Music", red in tooth and claw.
No one who listens to the Blues with any sense of objectivity can call them sad.
This music is always an assertion of life, its pains and its pleasures.
When Blind Willie Johnson sings about The Titanic (God Moves), it is from his perspective as a commentator on the vagaries and uncertainties of life.



\subsection{The linguistic procedures that typify African music}

It is still possible to track down some recorded examples of African music, which has not been adulterated by exposure to Western music.
Happy Hunting!

What the reader will find is a language that contains many of the elements that are now fundamental to present day African American music.

These elements are:

\begin{itemize}
    \item Call and Response - Here, the lead voice (voices) prompts the "chorus"  to a response with a lead line.
        There is not necessarily any melodic connection between the two.
    \item Layering - African music proceeds on several layers.
        (This is NOT counterpoint in the European sense).
    \item Repetition - Each layer is built up of a repeated pattern.
        Patterns can be of varying lengths, and are not necessarily related to the patterns that occur in other layers.
    \item Soloing - At various points in the music, a singer, drummer or whatever shows off a particular skill or musical trick.
    \item Ululation - This is a kind of vocal howling and wailing, used to simulate and stimulate great emotion.
\end{itemize}

As with all languages the fundamental principals on which they are built are simple.

THE MANIFESTATION OF THESE ATTRIBUTES IN THE BLUES

Listen to any Blues, R\&B or Gospel recording, and you will notice ( All now contained within bar lines):

\begin{itemize}
    \item Call and Response - Singer sings lead line and the guitar responds.
    \item Layering - depends on the number of musicians and elements involved, but is obvious.
        Each element has its own place.
    \item Repetition - Everywhere.
        The simulateous cyclic nature is there for all to hear: The 12 bar cycle, the riffs (1 or 2 bar usually), the rhythm section.
    \item Ululation - Is also found everywhere.
        In both the vocal and instrumental layers non specific pitches, sounds for made their own sake are part of the very essence of this music.
\end{itemize}

\subsection{Linguistic and formal attributes of The Blues}

These attributes are what typify the language of the Blues.
You will be able to recognise what follows in all the examples of this music that you come across.
Naturally, different regional, personal and historical styles will differ from each other, as they do in other musical languages, but the fundamental attributes and procedures remain constant.
Indeed, it is this very fact that defines the Blues as the progenitor of the African American musical language that is, in its numerous manifestations, the dominant force in today's world.

The first and most important formal attribute is that The Blues are cyclic.
The aforementioned layers are in themselves cyclic structures, that move in parallel, are synchronous, but which are harmonically and melodically independent.

A cycle is, of course, just a repeated pattern.
What you get in The Blues are repeating patterns layered upon repeating patterns.
Patterns can be, and are, of various lengths.
Some elements may occur sporadically and seemingly appear to be random, but on repeated listenings you will discover that this music is highly organised from a structural perspective.
The improvisatory elements work because of this organisation.

These patterns/cycles are as follows:

\begin{itemize}
    \item The harmonic 12 bar cycle: using the Western Primary Triads.
    \item The explicit rhythmic cycles that  express the pulse and reinforce the bar lines:
        \begin{enumerate}
            \item Each drum/percussion sound has its own individually constructed pattern or group of patterns.
            \item The "rhythm guitar" and "boogie woogie" elements are, again, independently structured.
            \item Then there is the bass line's "riffing" element.
            \item The lead instument also "riffs".
            \item The chorus' hooks are also, effectively, rhythmic cycles.
        \end{enumerate}
\end{itemize}

In addition to these regular, rhythmic and harmonic cycles, there are the independent and more flexible melodic patterns.

\begin{itemize}
    \item The lead vocal, though flexible and varied, is essentially cyclic.
    \item The vocal and instrumental "responses" are also cyclic and pattern based.
    \item The solos, built either by varying the melody or on developing the scales, chords and riffs, is also a part of a cyclic scheme.
\end{itemize}

In the case of solo performers, most of the above are compressed into a more compacted, though no less sophisticated, series of layers.
\end{document}

\documentclass{article}

\title{Teaching Resources}
\date{}

\begin{document}

\maketitle
\tableofcontents

\newpage
\section{Thoughts of Chairman Owen}

\subsection{Introduction}

A little red book of sayings...

When I first started teaching, longer ago than I care to remember, when Chairman Mao was still around, I wrote a one line piece of advice in a pupil's notebook.
I called it, and later examples, ``The Thoughts of Chairman Owen''.
I explained to this now long forgotten pupil that Mao got the idea from me.
The ``T.O.C.O.'' have remained a joke that I share with my pupils.
Alas, I am now the only one who understands it, as none of my pupils have ever heard of Mao!
When I tell them who he was, they politely smile to keep me happy in my old age.

These one-liners are all very useful and and to the point.
Any pupil who follows them conscientiously will feel the benefit!

I still use them all the time.

\subsection{TOCO}

\begin{itemize}
    \item Practise is the art of intelligent repetition.
    \item A good player never does what he/she does not intend doing.
    \item Never assume anything is correct.
    \item Mistakes don't always sound wrong, but in time always sound right.
    \item Speed is not a virtue, accuracy is.
    \item When doing technical exercises, play as fast as you can, not as fast as you can't.
    \item Slow practise is thinking time not action time.
    \item No one ever became a fast runner by walking everywhere.
    \item There is a speed at which anybody can play anything correctly.
    \item A good player never does what he or she does not intend doing.
    \item All mistakes are caused.
    \item If you don't make mistakes you can't fail.
    \item What you do not notice you cannot practise.
    \item Every note must be consciously articulated.
    \item Play the piano, don't press it.
    \item Good practise stops a ten hour problem from becoming an eleven hour problem.
    \item The nature of a pupil's work dictates the kind of lesson the pupil will have.
    \item Keep things simple : play what is on the page and what is implied by what is on the page.
    \item The more you look, the more you see.
    \item When your mouth is open, your ears are closed.
\end{itemize}

\textbf{Happy practising!}

\newpage
\section{Harmony in the Style of Bach}

Rules and Guidelines for Writing Four-Part Harmony in the Style of Bach.

\subsection{Introduction}

The study of Bach’s Harmony has, along with 16\textsuperscript{th}.
Century vocal polyphony, long provided the basis for understanding Classical Music in all of the World's leading Conservatoires.
Beethoven’s studies of Bach’s Chorales are well documented.

The following notes are only intended to provide an introductory overview, and students are encouraged to study the Chorales alongside these notes in order to gain a thorough understanding of the subject.

Harmonizing Chorales in the style of Bach is difficult!
It takes a long time to become proficient.
The effort is worthwhile and very rewarding.

These notes are somewhat different from those which can be garnered from traditional Harmony Textbooks, in that they not only tell you what you can and cannot do, but also tell you what you should and should not do.
Merely ``following the rules'' is \textbf{not enough}!
You must learn how to exercise your own creative judgement just as Bach would have done.
Quite a challenge!

Note well therefore, that my use of the words must, must not, and should are very important.

Finally, in the course of the various examinations you will be sitting, you will be required to demonstrate your knowledge of particular chord progressions.
Whilst these may not be particularly desirable in the context of Bach, you must do them \emph{if you want to pass}.

\textbf{Everything you write must be explicable according to the following:}

\subsection{Rules and guidelines}

\subsubsection{Chords and cadences}

Cadences:

\begin{center}
\begin{tabular}{||c|c|c||}
    \hline
    Cadence & Progression & Ratio \\
    \hline
    Perfect & V -- I (VII -- I) & 75\% \\
    Imperfect & Anything (Usually I, II, or IV) -- V & 25\% \\
    Plagal & IV -- I & Very rarely \\
    Interrupted & V -- VI & Very rarely \\
    \hline
\end{tabular}
\end{center}

\begin{itemize}
    \item Perfect V -- I (VII -- I).
        You should use these cadences most of the time
    \item Imperfect anything -- V.
        (Usually I,II,IV -- V.)
    \item Plagal IV -- I.
        Used very rarely
    \item Interrupted V -- VI.
        Used very rarely.
    \item A cadence must be used at the end of every phrase
    \item The penultimate chord may be a seventh.
        The last chord in a phrase must not be a seventh.
    \item Cadences should be in root position.
        The penultimate chord can be a first inversion.
        The last chord in a phrase must be in root position.
\end{itemize}
Chords:
\begin{itemize}
    \item Never use chord III.
    \item Chords I,II,IV,V,VI and VII are allowed, as are II/7, V/7, and VII/7.
    \item Chord IV/7 is allowed ,but is best avoided.
    \item Every chord should contain every note.
        If in difficulty you may omit the 5\textsuperscript{th}.
    \item Never omit the 3\textsuperscript{rd}.
    \item Never double the leading note.
    \item Double the root or the 5\textsuperscript{th}.
        Ratio 90\% -- 10\%.
    \item Never double the major 3\textsuperscript{rd}.
        (A few special conditions allow an exception to this rule.)
        Avoid doubling the minor 3\textsuperscript{rd} until you are sure of where this is allowed.
    \item Chords must be used in root position and 1\textsuperscript{st} inversion.
    \item 2\textsuperscript{nd} inversions are allowed only in the following progressions:

        \begin{tabular}{lclcll}
            Ic  & -- & V  & -- & I   & (Perfect Cadence);\\
            I   & -- & Vc & -- & Ib; &\\
            Ib  & -- & Vc & -- & I;  &\\
            IV  & -- & Ic & -- & IVb;&\\
            IVb & -- & Ic & -- & IV. &\\
        \end{tabular}
\end{itemize}

The first is known as a Cadential 6/4, and is a very good progression.
The others are known as Passing 6/4s and are usually avoided by Bach.\footnotemark
    \footnotetext{These are very important examination progressions and should be used when an opportunity presents itself.}
Seventh chords are 4 note chords and any inversion is permissible, but you should not use the 2\textsuperscript{nd} inversion.

\subsubsection{Rules of progression}

\begin{itemize}
    \item Consecutive Perfect 5\textsuperscript{th}s and Octaves are forbidden between two parts when they are moving in parallel.
        (N.B. This does not apply to static parts)
    \item Never use the same chord over a bar line except as an upbeat(anacrusis) at the start of a phrase.
    \item The leading note must go to tonic except at a Perfect Cadence where it can drop directly to the 5\textsuperscript{th}.
        This resolution may be carried over to another part.
    \item 7\textsuperscript{th}s must resolve downwards by step.
    \item Where there is conflict between applying the leading note and 7\textsuperscript{th} rules, the resolution of the 7\textsuperscript{th} must prevail.
    \item Never make a melodic leap of an augmented interval.
    \item Diminished intervals are allowed providing that it is resolved inwards.
    \item Avoid the major 6\textsuperscript{th} as an upward leap.
\end{itemize}

\subsubsection{Guidelines for good harmonic progressions}

\begin{itemize}
    \item Use Primary Chords as much as possible.

    \begin{tabular}{||p{0.2\linewidth}|p{0.2\linewidth}||}
        \hline
        Chord & Ratio\\
        \hline
        I & 40\%\\
        V & 30\%\\
        IV & 15\%\\
        II & 10\%\\
        VI & 5\%\\
        \hline
    \end{tabular}

\item For strong progressions you should try to aim for root movements of a 5\textsuperscript{th}, i.e. I -- V -- I --  IV -- I -- V -- II -- V -- I etc.\footnotemark
        \footnotetext{II -- V -- I is a very good progression for a Perfect Cadence, and is better than IV -- V -- I.}
    \item Try to keep the bass part moving in contrary motion to the upper parts as much as possible.
    \item Keep the inner parts (alto and tenor) static.
    \item Keep the tenor part high.
    \item Keep all parts moving smoothly.
        Avoid successive leaps in the same direction.
    \item Try to keep the top parts within the range of a 10\textsuperscript{th}.
    \item Chords should move at a regular pace (Harmonic Rhythm), and every note that falls on a beat should be harmonised.
\end{itemize}

\subsubsection{Modulation}

You should modulate freely and without inhibition.
In Bach’s music, interest is obtained through modulation.

\begin{itemize}
    \item You must only modulate to closely related keys; i.e. those which are no more than one sharp or one flat away from the home key.
        (Home key $\pm$5.)
    \item The whole Chorale must begin and end in the same key.
        All the other phrases can be ended in any of the related keys.
    \item You may modulate several times in the same phrase.
    \item You may modulate without any preparation and at will, but must demonstrate that a modulation has occurred.
        The progression V -- I in the new key is all you need!)
\end{itemize}

In text books, modulation is dealt with in a frequently contradictory manner.
It seems to be very difficult.
This however is not the case.
In the style of Bach, the much written about ``Pivot Chord'' does not exist.
This simple fact lies behind the confusion that this topic causes.
``Pivot Chords'' do exist in later musical styles as a means of modulating to the more remote keys smoothly, but that is another story!

\subsubsection{Inessential notes}

There are three types of Inessential Note: The passing note, the auxiliary note\footnotemark and the suspension.

    \footnotetext{Passing notes and auxiliary notes can give rise to consecutive 5\textsuperscript{th}s and octaves, but cannot save you from consecutive 5\textsuperscript{th}s and octaves.}

\begin{itemize}
    \item[] Passing Notes:
        \begin{itemize}
            \item They can occur off the beat ascending and descending and on the beat descending only.
            \item Two successive passing notes can occur in the bass part descending only.
        \end{itemize}
    \item[] Auxiliary Notes:
        \begin{itemize}
            \item Upper and lower can only occur off the beat.
        \end{itemize}
    \item[] Suspensions:
        \begin{itemize}
            \item Must resolve downwards by step.
            \item Try to look for a good close clash when you use a suspension.\footnotemark
            \item They usually work best in the inner parts, though Bach makes them sound good in any part.
        \end{itemize}
\end{itemize}

    \footnotetext{The 4-3 suspension is the strongest and most striking, as it resolves onto a note that is not currently sounding.
        9-8 suspensions can sound good, the 6-5 is the least likely to work well.}

\subsubsection{Tierce de Picardy}

In the minor key, end on the major chord.
This can happen in other phrases as well.

\subsubsection{Conclusion}

Use The Bach Chorales as your text book, but be warned, at one time or another, he will have broken every single rule and guideline which I have outlined above!
\textbf{He is allowed.
You are not!}

\newpage
\section{Counterpoint}

Rules and Guidelines on Counterpoint.

\subsection{Introduction}

These notes assume that you have a comprehensive knowledge of the rudiments of music theory, and that you have a thorough grounding in the harmonic style of J.S.Bach.

\subsubsection{Counterpoint and Harmony}

The two disciplines are distinct but not mutually exclusive.

When writing Harmony, the principal objective is to create satisfying sequences of block chords of a mood and character which is appropriate to the style of the given melodic or bass part/line.

When writing Counterpoint, the principal objective is to create simultaneous progressions of different voices which, though harmonically interdependent, stimulate the listener to hear and appreciate each individual melodic line.

It is safe to say, therefore, that counterpoint is the more intellectually demanding of the two disciplines for both writer and listener.

\subsection{Rules}

The rules of counterpoint are essentially the same as the rules of harmony.
They depend on the style you wish to re-create.
Therefore, in the style of Bach, the rules of melodic progression and the rules (and guidelines ) which are pertinent to the style of Bach apply in counterpoint as well.

\subsubsection{General guidelines and specific techniques}

In order to write counterpoint successfully you should thoroughly acquaint yourself with the following information, and patiently acquire and (hopefully) master each of the techniques described below.

To write good counterpoint one must learn the art of thinking strategically, so as to fully exploit (without breaking the rules) the potential creative possibilities which are inherent in the techniques of counterpoint.

These techniques are:
\begin{itemize}
    \item Canon
    \item Imitation
    \item Inversion
    \item Augmentation
    \item Diminution
    \item Sequence
    \item Fragmentation
    \item Melodic Variation
    \item Rhythmic Variation
\end{itemize}

\subsubsection{Step 1: Writing a good melody}

You must learn, first, how to write a good melody.

\begin{enumerate}
    \item It must have a strong personality: characterful intervals, a definite rhythmic propulsion and a clear sense of harmonic direction.
    \item It must not, ever, be overly complex.
    \item It should not cover much more than an octave in pitch.
    \item It should contain enough space to allow counter melodies the chance to register on the ear.
    \item It should be kept short enough to be manageable
    \item It should (very difficult!) contain the potential for a certain amount of harmonic ambiguity.
        (This contradicts 1, but the reasons will become obvious later).
\end{enumerate}

\subsubsection{Step 2: Writing a good counter-melody}

\textbf{Advice:}
As in step 1, and additionally:

\begin{enumerate}
    \item It must complement the main melody, and not detract from its character.
    \item It must have its own individual character and personality.
    \item It must be harmonically compatible with the first melody.
    \item It must not be so different as to jeopardise musical cohesion.
        (Very difficult indeed!)
\end{enumerate}

\subsection{Species counterpoint}

Species Counterpoint is a very difficult to acquire, but invaluable, skill, which has long formed the basis of Contrapuntal Study in many of the World’s leading teaching establishments.

Practising good note manipulation skills through the study of Species Counterpoint.

Writing against a ``Cantus Firmus'', you should become proficient in each species, first in 2 part writing, then 3, and so on.

\begin{tabular}{p{0.2\linewidth}p{0.65\linewidth}}
    1\textsuperscript{st} Species & Semibreve against semibreve.\\
    2\textsuperscript{nd} Species & Minims against semibreves.\\
    3\textsuperscript{rd} Species & Crotchets against semibreves.\\
    4\textsuperscript{th} Species & Syncopated tied minims against semibreves.\footnotemark\\
    5\textsuperscript{th} Species & A free mix of Species 1 - 4.\\\\
\end{tabular}

    \footnotetext{First note is a minim, last is a semibreve.}

For those from the U.S.A.:\\

\begin{tabular}{p{0.2\linewidth}p{0.65\linewidth}}
    Semibreve & Whole note\\
    Minim & Half note\\
    Crotchet & Quarter note\\
    Quaver & Eighth note\\
    Semiquaver & Sixteenth note\\
\end{tabular}

\subsubsection{Melodic rules}
To gain the most value from the study of species counterpoint the rules are necessarily very restrictive.
To quote my teacher, Nadia Boulanger: `Everything is forbidden.'

\begin{enumerate}
    \item All melodic leaps are allowed except:
        \begin{enumerate}
            \item Augmented Intervals.
            \item Any interval wider than a Perfect 5\textsuperscript{th}.
        \end{enumerate}
        Notes:
        \begin{itemize}
            \item You may leap a minor 6\textsuperscript{th} upwards, but must resolve inwards.
            \item You may leap an octave, but only once per cantus firmus.
            \item You may leap a diminished interval, provided that you resolve it inwards.
        \end{itemize}
    \item You may not have more than two consecutive leaps in the same direction.
    \item You may not have any sequential melodic patterns.
\end{enumerate}

\subsubsection{Intervalic rules governing the relationship between two parts (1\textsuperscript{st} species)}

Allowed:
\begin{itemize}
    \item 3\textsuperscript{rds} and 6\textsuperscript{th}s Major and Minor.
    \item Perfect 5\textsuperscript{th}s
    \item Octaves
    \item Unisons
    \item 1\textsuperscript{st} inversion of a Diminished 5\textsuperscript{th}.
\end{itemize}

Not allowed:
\begin{itemize}
    \item Everything else
\end{itemize}

\subsubsection{Rules concerning the progression of two parts (all species)}

\begin{itemize}
    \item No consecutive 5\textsuperscript{th}s and octaves between bars;
    \item No melodic repetition;
    \item Avoid more than two parallel movements and never do more than three;\footnotemark
    \item Never have a unison or octave on the first beat of a bar except at the start and finish;
    \item It is wise to avoid the Perfect 5\textsuperscript{th} at the beginning of a bar.
\end{itemize}

\footnotetext{Contrary Motion is a fundamental ideal in Western/European Music and you must consistently strive to achieve it.
Indirect movement (oblique) is O.K.}

\subsubsection{Melodic rules (2\textsuperscript{nd}, 3\textsuperscript{rd}, 4\textsuperscript{th} and 5\textsuperscript{th} species)}

Passing and Auxiliary notes are allowed.
They must however only occur on ``weak'' beats.
That is, in a position where the two Harmony Notes which are being linked are in a rhythmically superior position.

\subsection{No pain, no gain}

Each Cantus Firmus is to be done 5 or more times.
Each example to be different from all the others!

\newpage
\section{Piano}

\subsection{Part 1}

Some thoughts on piano teaching and how to acquire a sound physical technique.
I have spent a lot of the last 25 years teaching the piano, with some success.
Below are some thoughts and observations about the art of teaching which may be of some help to other teachers.
While anyone is welcome to use these notes, please respect my copyright and acknowledge this resource where appropriate.

This is in no way meant to be a comprehensive or thorough treatise, rather it is an outline of my philosophy of teaching with some helpful pointers and advice.

The great Russian teacher Heinrich Neuhaus’ book ``The Art Of Piano Playing'' is the definitive work on the subject and it should be on every music teachers book shelf!
My contribution, therefore, makes no attempt to enlarge upon Neuhaus’ brilliant book.
I write from the less rarefied world of the ``shop floor''.

\subsubsection{Some introductory remarks concerning teaching in the UK}

(A Very Personal/Partisan View.)

Here in The UK, we continually labour under the culture of mediocrity that so characterises British classical musical life.
When someone from these islands does achieve distinction, it is usually ``in spite of the system rather than because of the system''!

The mentality of ``The Party Piece'' is pervasive at all levels of our music education system.
Music is not taught with any real sense of rigour.
Comparisons between the product of, say, Julliard, The Moscow and Paris Conservatories and our own UK institutions are stark and embarrassing, but only very rarely acknowledged!
It is a fact that although hundreds of thousands of people learn instruments in their childhood here in the in Britain, very few are able to play competently, if at all, in adulthood!
The investment, both personally and financially is not rewarded with results that are in any way commensurate to the effort expended.

I believe that the quality of teaching must improve at all levels, if our children are to compete on equal terms with those of the former Soviet Union, Europe, The Far East and The United States.

\subsubsection{The aims of teaching \& the responsibilities of the teacher}

\begin{enumerate}
    \item The first and principal task of teaching is to help enable pupils to achieve their full potential on their chosen instrument.
        So that they can enjoy the many pleasures and benefits of being able to play music well.\footnotemark

        \footnotetext{Almost everyone is capable of reaching a reasonable level of fluency with hard work and good teaching.}
    \item Secondly, to enable the talented pupil to reach levels of performance which are appropriate to their gifts.\footnotemark

        \footnotetext{This forms the nub of my approach.
        All teaching must be geared towards the gifted child.
        If this is done, even the average pupil will benefit, and those with obvious, latent, or not so obvious gifts, will not be missed or let down.}
\end{enumerate}

\textbf{Rule:} Assume from the start that the new pupil in front of you may be capable of becoming a great artist.\footnotemark
\newline
\footnotetext{Unfortunately, by the ``law of averages'' this is unlikely to be the case, but if you make this assumption, you are already in the position where you treat each and every pupil seriously and place upon yourself the moral and musical expectation of high standards and integrity.}

If you take the above as your starting point, you are already on the road to being a better teacher.
By definition, the great artist must acquire a formidable and comprehensive range of abilities: imaginative, technical, personal and moral.
Are you up to the task?
Do you, personally, know what it takes?
Do you have the necessary skills, insight, imagination and personal qualities to guide your pupils towards their goals?
\newline

\textbf{Rule:} If you cannot do something yourself you cannot teach it!
\newline

The above may seem over prescriptive and dogmatic, however, what is the alternative?
Art is nothing if it is not rooted in integrity and morality.
Quality is everything!
Without it we as teachers betray our pupils and demean ourselves.
Therefore, avoid creating the delusion of achievement, the comfort of mediocrity, the endless stream of pupils, who learn for a while, get a few Grades, then give up, only to be replaced by another hopeful face, for the cycle to be repeated : on and on and on.
Does this sound familiar?

Note : Of course, in our throw away age, some pupils will, inevitably, ``give up''.
It is my firm opinion, however, based on experience, that more will, and do, stay the course if the teaching is of a high standard, rigorously applied!

\subsubsection{Assessing musical talent}

What is musical talent?
Can it be quantified?
These questions have exercised me for a long time.

I have come to the conclusion that musical talent is an amalgam of gifts.
The more of these gifts that are possessed, the greater the talent.
A talented person should have (in no particular order):

\begin{enumerate}
    \item A good ear and sense of pitch.
    \item Good rhythm.
    \item Intelligence.
    \item Physical gifts (Good hands, strength Etc.)
    \item Good hand to eye co-ordination.
    \item Imagination.
    \item The ability to work hard, unaided.
    \item A sense of adventure.
    \item An inquisitive mind.
    \item Emotional sensitivity and flexibility.
    \item Good/strong character.
    \item Individuality.
    \item Desire and ambition.
    \item Good background (This IS a gift!)
    \item Love music!
\end{enumerate}

Individually, all of the above are measurable.
When combined, in abundance, the possessor of these gifts can be said to ``have talent''.\footnotemark
After careful teaching I have on several occasions been surprised and delighted by a pupil suddenly becoming talented.

    \footnotetext{People develop at differing rates.
    Therefore the teacher must not judge pupils prematurely, lest they make a mistake and a talent goes unnoticed.
    Many factors effect development including: environment, upbringing and general education.
    Some pupils (even very gifted ones) can take a long time in becoming confident in their own abilities.}

\subsubsection{Basic principles of piano playing and teaching}

Playing the piano is at root easy!

Playing a note requires virtually no effort!

Understanding musical notation and relating it to its position on the keyboard is a simple matter for all but the most unintelligent!

The basic principles of tone production are easily explained - the harder a key is struck the louder the sound.

Knowing which parts of the locomotor mechanism are used to produce a sound, and where to do what, is simply a matter of common sense, methodically and intelligently applied.

Problems only start to develop when confusion enters the arena.
Keep things simple, and always explain what you teach and why, at every single stage of the process.\footnotemark

    \footnotetext{Never teach by rote!
    (Human beings are not to be confused with Parrots and Performing Seals!) If you do, you are storing up problems for the future.
    These problems are often incurable.}

Never expect a pupil to advance at a pace which is faster than what they are physically, emotionally and intellectually capable of!

\paragraph{At this point: a few tips (``Words of Wisdom'')\newline\newline}

My pupils know these as ''The Thoughts of Chairman Owen''.
From now on referred to as TOCO.

\begin{enumerate}
    \item A teacher’s job is to enable the pupil to become independent of the teacher, not dependent upon the teacher.
    \item No note is difficult to play, in between the notes are where all difficulties reside.
    \item There are two kinds of note: Finger notes and Arm notes.
        All notes which are not legato are Arm notes.\footnotemark

        \footnotetext{This crystal clear advice is slowly allowed to erode, and develop, in the light of experience, and the requirement to produce differing qualities of tone when the pupil reaches a more advanced stage.
        (Children and beginners alike need statements to be in Black and White, understanding shades of grey comes with time.)}

    \item What you do not notice, you cannot practise.
    \item Never assume anything is correct.
        Mistakes do not always sound wrong and in time come to sound right.
    \item Practise is the Art of Intelligent Repetition.\footnotemark

        \footnotetext{Practising and the techniques involved are dealt with later on.}

    \item The purpose of practise is getting things right, solving problems.\footnotemark

        \footnotetext{A mistake is a mistake is a mistake, and cannot be improved upon.}

\end{enumerate}

\subsubsection{Physical technique}

Physical Technique alone is, needless to say,worthless.\footnotemark
Poor technique places the performer under a handicap.
What they imagine cannot be realised.
Talent and insight without the necessary physical wherewithall is fatally compromised and inhibited.

    \footnotetext{Physical Technique is, of course, not an end in itself.
    It is a means to an end, namely: to eliminate the barrier between the printed score, the performance and the audience.}

There is no such thing as a natural technique.
So called natural technique is merely facility (often mindless) and has its natural limitations.
Technique is acquired through hard work and understanding.
Its potential for development is, for most people, essentially limitless, until the physical deterioration that comes with age sets in.

The Elements of Technique are :
\begin{enumerate}
    \item Speed;
    \item Stamina;
    \item Strength;
    \item Control.
\end{enumerate}

These can all be acquired without problems.
All that is needed is the aforementioned hard work, understanding and patience.

\subsubsection{How to acquire a good physical technique}

Technique is acquired in stages:

After first acknowledging three self-evident truths.

\begin{enumerate}
    \item The piano is simply a sound producing machine.
        It is an inanimate object.
        It only responds to physical action.\footnotemark

        \footnotetext{Musical elbows, tortured facial expressions, swaying bodies and the like, cut no ice with the average Steinway.
            All that stuff is strictly showbiz!}

    \item The human body has some inherent ``design faults'' which must be understood and overcome if the piano is to be played successfully.\footnotemark

        \footnotetext{Luckily, the design of the keyboard meets us half-way, and in time the student will come, indeed must come, to feel that the piano is a natural extension of his/her body and thus by extension, their mind and heart.}

    \item Acquiring an ``easy'' relationship with the instrument is the vitally important goal towards which, all who study the piano must strive.\footnotemark

        \footnotetext{One of the sure signs of poor teaching, is the pianist who is not at ease with the instrument.
        These pupils seem, always, to be in awe of the piano.
        It is the enemy rather than a friend!}

\end{enumerate}

\subsubsection{The basic principles}

\begin{tabular}{p{0.2\linewidth}p{0.65\linewidth}}
    The Fingers: & The job of the fingers is to play (strike) legato notes, uninhibited by natural weakness.\\
    The Arms: & The job of the arms is to play (Drop with controlled weight) all non-legato notes, take the hand along the keyboard, and provide a stable (relaxed\footnotemark) platform from which the hand can operate.\\
\end{tabular}

    \footnotetext{Relaxation is an often misunderstood concept.
    If one were ever to be completely relaxed, one would of course fall off the piano stool!
    Rather, relaxation is the art of using only as much force/tension as is absolutely necessary, and in not wasting energy through the inefficient use of the body.}

\subsubsection{Understanding the hand}

\begin{enumerate}
    \item Thumb needs attention - designed to work on a different plane to that required on the piano; designed for gross movement, it is not by nature sensitive.
    \item Finger two - well designed.
    \item Finger three - almost as good as finger two.
    \item Finger four - hopeless!
        Does not have its own independent mechanism.\footnotemark

        \footnotetext{Evolution did not have pianos in mind when our species set out on its journey to now.}

    \item Finger five - weak and unwilling to move.\footnotemark

        \footnotetext{Finger Five: Particular attention should be given to this finger.
    There is a tendency for it to not articulate.
    This is because it is weak and the joint is rarely, if ever, utilised naturally.
    Therefore the hand ``falls/leans'' towards it, in order to compensate for its inherent weakness.
    The result is that it is not encouraged to develop, and the pupil labours, permanently, under a handicap.}

\end{enumerate}

In addition, all our fingers are of a different length, have a tendency to collapse at the joints (thereby absorbing energy and using up precious time), and do not, naturally, use the full range of movement that the joints allow.
They need exercise to develop their potential.
\newline

Play on the tips of the fingers and the side of the thumb.
This is correct advice.
Why?

\begin{enumerate}
    \item The natural, relaxed, position of the hand when brought up to the keyboard will, by default, bring the tips of the fingers into contact with the keys.
    \item By playing on the tips of the fingers, the problem of different length fingers disappears.
    \item Playing on the tips of the fingers creates space for the thumb to pass under when playing a scale.
    \item Playing on the tips of the fingers, gives each finger room to move/articulate.
\end{enumerate}


\subsubsection{What exercises?}

There are two schools of thought with regards finger exercises.

One opinion is that exercises are unmusical and sterile; that they encourage a mechanistic approach in the pupil.
The people who support this point of view say that technique acquired through the learning of pieces integrates the mechanical with the musical.

My own view is that pieces provide inefficient training.
They were conceived as music, therefore, they do not present difficulties with enough intensity.
Real music was not designed to be graded according to technical difficulty.

Furthermore, to reduce a piece of music to the status of a ``study aid'' is to diminish its worth.
By the time a piece has been learnt, and all of its problems overcome and all of its technical lessons absorbed, love for the piece can be, and often is, destroyed.

Far better therefore, to isolate the purely mechanical.\footnotemark
    \footnote{I am not dogmatic about this.
    It is all a matter of judgement.}
    \footnotetext{One of the great joys of playing a musical instrument, is the physical and athletic exhilaration and joy which is an inherent in the activity.
    Playing fast is fun!}\pagebreak

\textbf{I think that the following exercises are as good as any:\footnotemark}

    \footnotetext{It is not the books you use that are important, but the way you use them.}

Scales, arpeggios, etc. of course - plus:\footnotemark

    \footnotetext{I dislike the Hanon-type exercises as they are long and repetitive for no good reason.}

\begin{itemize}
    \item Beginners:    Schmitt - Preparatory Finger Exercises;
    \item Intermediate: Dohnanyi - Essential Exercises;
    \item Advanced:     Brahms - 52 Exercises, and Philipp.
\end{itemize}

Studies, should be learnt in conjunction with the above exercises:

\begin{itemize}
    \item Beginners: Bartok - Mikrokosmos Books 1, 2 and 3;
    \item Intermediate: Czerny (Take your pick);
    \item Advanced: Chopin, Liszt and Debussy.
\end{itemize}

\subsubsection{Understanding the arm}

The arm makes two movements in piano playing.
The vertical and the horizontal.

\paragraph{The vertical movement of the arm.\newline\newline}

This is best understood by dividing the arm into three.
The upper arm, the forearm and the hand.
In order to lift and then drop with controlled weight the following must occur:

\begin{enumerate}
    \item The lift should always be in time: On a beat or the exact proportion of a beat.
        If this is not possible the note preceding the lift is staccato or detached.
    \item The drop should be smooth yet precise.
        The upper arm moves a bit, the elbow amplifies this movement into the forearm, then the hand strikes the keys with the wrist acting as a spring and shock absorber.
\end{enumerate}

Notes:
\begin{itemize}
    \item This weight transferred through the wrist is the piano’s ``tone control''.
    Sensitivity for the varying amounts of tension the wrist is capable of is one of, if not the, most difficult skills a pianist must master.
    \item It is anatomically efficient to have the wrist level or slightly below the level of the hand.
        High wrists are a sure sign of tension, and result in poor tone.
    \item It is impossible to bang (jar the piano mechanism) if one uses weight as opposed to force.
\end{itemize}

\paragraph{The horizontal movement of the arm.\newline\newline}

The arm must move sideways in order to take the hand to the desired location on the keyboard.

\begin{tabular}{p{0.2\linewidth}p{0.65\linewidth}}
    \\
    TOCO: & What you are not above you cannot play.\\
    \\
\end{tabular}

If you are not precisely above what you play, you must, by definition, apply some force to the drop.
In order to move accurately the arm should take/lead the hand and not follow it.
\newline

Notes:
\begin{itemize}
    \item It is very important to keep the relationship between the arm and hand in a constant, straight plane for as far along the keyboard as is practical
    \item (\textbf{Very Important}) It is inevitable that a piece of music may force the pianist out of these idealised positions (hand shape Etc.).
        These should be forced onto the pianist by circumstances and not anticipated.
\end{itemize}

This sideways shift must likewise occur on a beat or an exact proportion of a beat.\footnotemark

    \footnotetext{Everything a player does should be rhythmical.
    If this is not the case tension can and often does result.}

All of the above technique for the arm can be acquired through the study of Bartok’s Masterpiece Mikrokosmos in the early and intermediate stages,

\subsubsection{Finally, hand staccato}

When playing repeated notes, the technique used is speed dependent.
Slow, you play with the arm, then, as you accelerate the hand takes over playing in rhythmic groups, finally when the hand cannot go any faster, finger staccato (repeated notes ) takes over.

\subsection{Part 2: Style and interpretation: general principles}

\subsubsection{Introduction}

Part two of this series of articles about piano playing and teaching, deals with some generalities about how to approach a new work.
Specific composers are dealt with in later parts.

\begin{tabular}{p{0.2\linewidth}p{0.65\linewidth}}
    \\
    TOCO: & What you cannot imagine, you cannot practise.\\
    \\
\end{tabular}

The job of the pianist is to interpret the composition, to place before the listener/audience as accurate and faithful an account as is within the capability of the performer.
In other words, to play what is actually written in the score and what is implied by what is written.

\begin{tabular}{p{0.2\linewidth}p{0.65\linewidth}}
    \\
    TOCO: & The performer is the vehicle for the music.
    Music is not a vehicle for the performer.\\
    \\
\end{tabular}

This requires an attitude of humility, inquisitiveness, an understanding of style, musical theory and, of course, imagination.

\begin{tabular}{p{0.2\linewidth}p{0.65\linewidth}}
    \\
    TOCO: & The performer is the essential partner to the composer.
    In order to live, a piece must be played.\\
    \\
\end{tabular}

This is the essence of performance: The ``living'' partnership and relationship between the written page and the performer, where the performer has the power to reveal or conceal; enlighten or obscure; be honest or dishonest.

\subsubsection{Where to start}

\begin{quote}
    \textit{In reading what follows, remember to trust your intuitive and spontaneous feelings, as they form the bedrock of any performance.}
\end{quote}

First we must assume that the pianist is adequately equipped to meet all eventualities:

\begin{itemize}
    \item The pianist must have a good physical technique.\footnotemark
    \item The pianist must be theoretically competent.
    \item The pianist must be artistically literate.
\end{itemize}

    \footnotetext{We must assume that the work to be studied poses no technical threat, and is not too difficult.}

First Steps ( The below may seem blindingly obvious, but in my experience, these steps are usually not followed.)

\begin{enumerate}
    \item Who composed the work about to be studied?\footnotemark

        \footnotetext{When answering this apparently innocuous question, the student (we are all students) has immediately got to know the answers to many other questions.
        For example: Are the markings on the score editorial or the composer’s own?
        This then leads to the following questions.}

    \item What period does the work belong to, and what are the implications of knowing this?
    \item What are the particular linguistic/stylistic characteristics of this composer’s music at this particular point in his development?
    \item What is the nationality/temperament/personality of this composer, and will this have any bearing on your approach?
    \item What is the piece called?
        Does its name have any particular historical or stylistic significance?
\end{enumerate}

The above gets you started on the next stage: namely deciding with a reasonable degree of certainty how to approach learning the notes and achieving an acceptably accurate initial interpretation.\\

Once the notes are known (and, indeed, during the note learning stage), the answers to the above questions will gradually get you moving towards gaining your own image of the piece/work.

\subsubsection{Next steps}

\begin{quote}
    \textit{Now the fun starts.}
\end{quote}

Ask yourself the following question: If we assume that a great pianist can sight read virtually anything and give an accurate and stylistically well informed performance straight off, \textbf{what do they do for five hours a day?}\\

\textbf{Answer:}
They are taking these next steps.\footnotemark

    \footnotetext{All these questions/challenges, and your success in dealing with them are a test of your integrity and insight.
    It is the measure of your artistry.
    It is your sacred trust.
    This work is the most joyful part of music making.
    No definite answers, only infinite questions!}

\begin{enumerate}
    \item What does this particular piece mean?
    \item What is the composer saying, or trying to say?
    \item What should I be feeling?
    \item Is what I am feeling appropriate?
    \item Is what I am feeling/imagining, ``at one'' with the music?
    \item Am I ``getting it across''?
    \item Could I do it this way?
    \item What if I do it that way?
    \item Why, when I play this like ``so and so'', does his playing seem right, but mine does not?
    \item  Is the way that is right for me valid and justifiable?
    \item At what point can I present this as a public performance?
\end{enumerate}


\begin{tabular}{p{0.2\linewidth}p{0.65\linewidth}}
    \\
    TOCO: & The greatest composers ask the most questions.\\
    \\
\end{tabular}

*See Piano Part One

\subsection{Part 3: How to practice}

\subsubsection{Before You Start}

In order to achieve good results from your practise, the following pointers are to be noted:

\begin{enumerate}
    \item 100\% effort = 100\% reward.
        (95\% effort does not = 95\% reward.)
        Satisfactory  results are only achieved through maximum effort.
    \item Go straight to the problem in hand, do not prevaricate.
    \item Good  working methods and techniques, allied to 100\% concentration will not make a 10 hour problem into a 9 hour problem, rather, they will prevent it from becoming a 12 hour problem, or worse!
    \item ``Patience IS a virtue''.
\end{enumerate}

The above points add up, in short, to this: Your attitude when working is all important, and no practice session should be undertaken unless you are in the right frame of mind.

\subsubsection{Goals}

The purpose of practise is simple.
You practise in order to achieve a particular goal.
This might be:

\begin{enumerate}
    \item To master a particular passage.
    \item To reach a good interpretation.
    \item To develop a particular skill.
\end{enumerate}
and so on...

Individually and in combination, achieving these goals contributes to your overall development as a performer and musician.
The more goals that are achieved, the further you are along the road that we all are on.
Although this road is endless, and perfection is unattainable, the reward is in the journey.
This is what makes music such a rewarding lifetime's activity.

\subsubsection{Practise: A Methodology}

\paragraph{Outline:} (Dealt with in depth below.)
\begin{itemize}
    \item Identify the problem.
    \item Isolate the problem.
    \item Analyse the problem.
    \item Devise an approach and working method for dealing with the problem.
    \item Follow the method through, until problem is solved.
    \item Integrate the problem back into its context.
    \item Go on to next problem........................
\end{itemize}

Obvious isn't it?
Easy as well.

\paragraph{So, what can go wrong and why?}

Firstly:

\begin{itemize}
    \item Lack of technical comprehension.
    \item Lack of musical comprehension.
    \item Shortcomings with regards physical technique.
    \item Shortcomings with regards musical technique and musicianship.
\end{itemize}

Then there are personal shortcomings:

\begin{itemize}
    \item Poor concentration.
    \item Lack of determination.
\end{itemize}

And so on...

\subsubsection{Identifying a problem}

\begin{tabular}{p{0.2\linewidth}p{0.65\linewidth}}
    \\
    TOCO: & What you do not notice, you cannot practise.\\
    \\
\end{tabular}

This is quite straightforward.
Let's face it, why practise anything if you have no problems?

Aside from the most obvious of mistakes, the failure to identify problems often arises from a shortfall in that most important of qualities: the capacity for self criticism.
Self-delusion is, I sometimes think, the natural state of mind for most people.)\footnotemark
    \footnotetext{Misread notes are always caused by carelessness and are inexcusable.
    Wrong rhythms are likewise caused by a lack of care.}
Other failures to identify problems are usually the result of tackling a piece that is too difficult and that you/the learner is ``out of your depth''.\footnotemark

    \footnotetext{Teachers should not give pupils work that is too difficult.
    If the pupil needs constant, phrase by phrase, passage by passage guidance, the piece is too hard.}

In this article, though, we are dealing with genuine difficulties, which can be defined thus:

\begin{itemize}
    \item Technically complex passages and musical/interpretative difficulties.
    \item In the case of technical difficulty, your locomotor mechanism, ear and even eyes, will tell you that all is not well, in even the most subtle situations.
    \item In the matter of interpretation, your mind, ear and feelings recognise the shortcomings.
\end{itemize}

The next stage is where the action starts!

\subsubsection{Isolating the problem}

This step is not quite as straightforward as it might first appear.

A series of wrong notes, or a clumsy leap, or a failure to project a melody, may only be a symptom of another underlying difficulty.

For example:

The failure to execute a right hand run may, in fact be caused by the left hand being poor at, say, the trill that it is having to play.
The poorly played trill might not be so obvious, but may be at the root of the problem.

It is important, therefore, that you isolate the correct problem.\footnotemark
\footnotetext{The process of isolating problems is closely linked to the next step.}
If you don't get this right, you will end up working at the wrong thing and you will waste time.

Experience is very important at this stage of the process.
Knowledge accumulated  from previous practise sessions, and the general awareness that comes with the aforementioned experience,in time, makes this stage increasingly straightforward.

\subsubsection{Analysing The Problem}

Problems can be divided into two categories:

\begin{enumerate}
    \item Difficulties which are specific to the piece being studied.
    \item General difficulties.
\end{enumerate}

When analysing the problem that you have isolated, ask the following questions:

\begin{enumerate}
    \item Is there an inherent weakness which is causing the difficulty?
    \item Is this a ``one off''?
\end{enumerate}

If it is a problem which is often encountered (a general difficulty), you should incorporate an exercise, or series of exercises, into your daily technical routine.\footnotemark
    \footnotetext{The deeper your comprehension of the principles of physical technique, and the greater your musical/theoretical and stylistic knowledge, the more effective your routines will be.
    See my first two articles for information about technique and interpretation.}
In time, this type of problem will then not recur, and you will learn pieces faster.

If the difficulty is a specific one, you need to work out its exact nature.
Then you can devise a routine or series of routines which you can follow in order to overcome it.


\subsubsection{Devising An Approach}

This is where you must be intelligent and, above all, imaginative.

To avoid making practise a dull and sterile experience, you need to devise fresh and creative,step by step, routines.
Avoid a merely mechanistic, ``by the numbers'' approach.

Treat your difficulty as a challenge to be overcome, not a chore to be endured.

Call upon your knowledge and understanding to construct your routines.

Invent new procedures.

Cut out the irrelevant.

\subsubsection{Following The Method Through}

\begin{tabular}{p{0.2\linewidth}p{0.65\linewidth}}
    \\
    TOCO: & Practise is the Art of Intelligent Repetition\\
    \\
\end{tabular}

\begin{itemize}
    \item Rigorously follow your routines through.
    \item Repeat each step as often as necessary, until it is thoroughly mastered.
    \item Do not proceed to the next step until the previous one is easy.
    \item Don't cheat.
        Stick to the task in hand until the end.
\end{itemize}

If, after following all of the above,you have been successful all well and good.
Proceed to the next step.
If, however, after a reasonable amount of  time, you have not been successful, reassess the position.
Ask yourself these questions:

\begin{enumerate}
    \item Was I really trying my hardest?
    \item Was I practising the right thing?
    \item Was my analysis correct?
    \item Were my routines appropriate?
\end{enumerate}

If the answer to any of the above is no: Start the whole process again, and learn from your mistakes!

Integrating The (Now Solved) Problem Into Its Context

Play from somewhere before the passage to somewhere beyond several times.
Get used to it no longer being a problem.

Congratulate yourself on a job well done, and go and have a cup of tea.

Then it's onto the next bit!

\subsection{Part 4: Bach}

Some Thoughts On The Performance of Bach's Keyboard Music.
It is hoped that this short article will introduce you to a more open-minded approach to the study of Bach's music than is the norm.
It is not intended to be, in any way, a complete course of instruction.
There are many excellent books on the subject, which you are encouraged to read.

Apologies, in advance, for any offence given to ``The Early Music Brigade''.
Actually, some of my best friends are early music specialists!

\subsubsection{Introduction}

Playing Bach's music on the piano is, in my opinion, the most creative, rewarding and enjoyable activity for any pianist to engage in.
In no other composer's music will you find so much scope for using your own judgement; so much scope for experimentation and so many potential interpretations.
The range of possibilities that each work presents are, in some ways, limitless.
Which performance is right; which approach is valid; which interpretation is best suited to a particular piece?
So many questions!
So many answers!

Indeed, this is probably the reason why comparatively few of the great pianists have the confidence to perform Bach in public.
(Fools go where angels fear to tread?)

Do not worry though, because, in truth, there are \textbf{no} definitive answers when playing Bach.

Comfort yourself with the thought that, of all the composers who ever lived, Bach's music can support/tolerate the full range of human expression.
An astonishing number of differing approaches are possible in virtually \textbf{every} piece.

\textbf{Bach truly is the greatest!}

\subsubsection{A word or two about authenticity}

It is my firm opinion that Bach's keyboard music positively benefits from being played on the piano!

Ignore the Early Music Fundamentalists!
They are wrong!
Follow their path and you will reach a joyless world of musical constipation.
In spite of all the academic research, very few authentic interpretations shed any new light on the glories and mysteries of this master's music.
Rather, the contribution of ``The Early Music Brigade'' is usually of the most mundane kind, namely: that when played ``authentically'' the music sounds different (thinner) from a performance given on modern instruments---Surprise Surprise!
The experience also shows us why  modern instruments were invented/developed!

\begin{tabular}{p{0.2\linewidth}p{0.65\linewidth}}
    \\
    TOCO: & If the harpsichord is such a great instrument, why did they bother inventing the piano?\\
    \\
\end{tabular}

The piano is a superior instument, in every way to its anaemic, rattly, predecessors.
It is a miracle of technology, a magic box: A percussion instrument that can (subjectively) sing!
Its range of possible sounds are infinite.
Its responsiveness to touch, unrivalled by any other keyboard instrument.

To deny oneself the opportunity of playing Bach's music on a modern piano is, quite simply, perverse!

\subsubsection{Ornaments}

Research into ornaments and their execution is often helpful, however, many of the conclusions that academics have reached are contradictory.

If you are going to play Bach on the piano, many suggested ornamental interpretations are in any case  impractical.
You will, therefore, have to use your own judgement and good sense when deciding whether or not to use a particular ornamental interpretation.

\subsubsection{General principles}

\begin{tabular}{p{0.2\linewidth}p{0.65\linewidth}}
    \\
    TOCO: & Bach's music is simply concerned with the joy of life and the miracle of our existence.
    (Bach composed Dance Music For Life.)\\
    \\
\end{tabular}

While this statement may not be empirically provable, what else could Bach's music be concerned with?
He was deeply religious, in a time of religious certainty.
Unlike later composers, such as Beethoven, Bach was predisposed to accept, on a spiritual level, life's meaning.
His fate was in God's safe keeping.

What is the consequence of this knowledge for the interpreter of Bach's music?

Simply this:

The swings of emotion, romance, angst and the other temporal passions that make up our daily lives, are of  little use when contemplating the universality of this master's musical expression.
You should, therefore, put aside your usual emotional palette, drawing upon which, while it aids the performance of other composers' work, will create a fog of incomprehension, when playing Bach.
Rather, consider the joyfulness of life and nature, and the miracle of our human existence.

Bach's music should be seen as an allegory for the universal truths.
Bach's music is about Goodness!

\begin{tabular}{p{0.2\linewidth}p{0.65\linewidth}}
    \\
    TOCO: & Bach's music is - ``Switched On'' Spirituality.\\
    \\
\end{tabular}

\textbf{Alone among all the great composers, bach used the tonal system as an explanation for the mystery of life and the universe.
His music means nothing outside itself.
The truth is in the notes and their joyous  relationship to one another - no more no less!}

\begin{tabular}{p{0.2\linewidth}p{0.65\linewidth}}
    \\
    TOCO: & Playing Bach's music is as close to seeing God as you will ever get.\\
    \\
\end{tabular}

\subsubsection{Let's get practical!}

When learning the notes, \textbf{do not} anticipate a particular interpretation!
Don't ``box yourself'' in.
Learn the notes in such a way, that you can change your view/performance of the piece, as it gradually reveals its particular nature to you, as an individual.

\begin{tabular}{p{0.2\linewidth}p{0.65\linewidth}}
    \\
    TOCO: & The more you look, the more you see.\\
    \\
\end{tabular}

Start with a blank page.

Ignore editorial markings.
These are of course, by definition, someone else's view and, no matter how scholarly, are a barrier to you acquiring your own understanding.
In time you will discover a suitable tempo, you will then be able to add your own articulation and dynamics to suit.
By starting with a blank page, your interpretative decisions will be positive and creative acts, motivated by a desire to reveal the music as you have come to see it, rather than a negation of someone else's view.

\subsubsection{Hints and tips}

Your task is to reveal the music's meaning through its inner workings.
Therefore, you must understand its construction: its key system and its form.

Identifying the main cadence points are an essential first step.
Knowing where these cadence points are, helps you understand the pieces formal structure.

Then you need to identify the thematic and motivic elements.
This is vital, so that you can create a consistent scheme for articulation and phrasing, in order to reveal the web of contrapuntal relationships and motivic interplay.

Start with the assumption that all quavers (eighth notes) in a predominantly semiquaver  (sixteenth notes) piece are detached.
In the fullness of time you will then arrive at a plan for slurring the various elements, that is, in your view, appropriate.
Remember, that the effectiveness of a particular articulation will be speed dependent.
In other words, what works at one tempo will not always work at another.

Gradually come to a view as to the overall nature of the piece.
What tempo should you take it at?
What dynamic range will you employ?
Is it more song than dance?
Should it ``rock''?
etc. etc.
\newline\newline
Before I go any further, I must debunk a few myths.

\subsubsection{Myths and mythology}

`Never use the sustaining pedal in Bach.' - Rubbish!
In Bach's day, there was no such thing as a sustaining pedal, therefore all notes were un-damped.
It is also safe to assume that the locations used for performances were very resonant (much more so than today's acoustically subdued environments), due to a lack of soft furnishings, wall paper etc.

The only rule, is that the sustaining pedal should not be used in such a way that is obscures the harmonic detail and contrapuntal flow.\\

`Because a harpsichord cannot crescendo nor should you.' - There is a grain of truth to this.
(But only a grain.)
This prescriptive advice supposes that crescendo and diminuendo were foreign to Bach.
This is, of course, nonsense.
But, because it was not within the capacity of the keyboard instruments of his day to gradually alter the dynamic, you should take into account  the fact that Bach imagined and composed his music without having this option available.
Remember though that you are playing on a piano, and to deliberately deny yourself access to its expressivity would be,as I have already observed, perverse!\\

`Use ``Terraced Dynamics''' - \textbf{Do not!} So called ``terraced dynamics'', (that is loud followed by soft, without intermediate dynamic gradations), are used to imitate the abrupt changes in tonal characteristics/timbre of which an organ or harpsichord are capable.
To do this in the course of a piece is wrong on almost all occasions.

There are several reasons for this:

\begin{enumerate}
    \item It is, usually, physically impractical to make rapid changes to the settings of a keyboard instrument, therefore, Bach would not/could not have done so himself.
    \item They cut across the contrapuntal flow.
        It is a fundamentally harmonic technique, and Bach is a contrapuntal composer.
\end{enumerate}

\subsubsection{Hints and tips (cont'd)}

Once the notes are learnt, the cadence points and musical material identified, the quavers detached, etc.
Ask these questions:

\begin{enumerate}
    \item What tempo is best suited to the musical content?
    \item What overall dynamic level suits this movement?
    \item What articulatory scheme will reveal the contrapuntal workings best?
\end{enumerate}

As previously stated, there will be no definite/final answers to these questions, but what you will find is this:

\begin{enumerate}
    \item The answer to one question will fundamentally effect your answer to the others.
    \item You will arrive at more than one possible interpretation.
        (Indeed, if you are really committed to the task you will arrive at several.)
\end{enumerate}

Take your time.
You will eventually arrive at an interpretation that seems to be just right.
When you do, realise that when you come to work at this particular piece again, you will almost certainly reach a different set of conclusions.

An Useful Exercise: Listen to the different recordings of the same works that Glen Gould has made.
You will notice that he has changed his mind fundamentally on several occasions.

\subsubsection{Stylistic considerations}

\begin{enumerate}
    \item Obviously, titled pieces (Sarabandes etc.) have to be performed within a comparatively narrow range of tempi.
    \item Untitled pieces, such as Preludes and Fugues, allow for a much, much  wider range of possible tempi.
    \item As Bach had no experience of the modern piano, dynamics should not be used as your primary means of interpretation.
    \item Your primary means of interpretation are:
        \begin{itemize}
            \item Speed.
            \item Articulation.
            \item Overall dynamic level.
                (One level per movement.)
            \item Pianistic expressivity.
                (Within your pre-defined dynamic parameters.)
        \end{itemize}
    \item Bach's music is predominantly, and fundamentally, rooted in dance.
    \item Bach's music is about the hierarchical nature of scales, the major-minor key system and its meaning.
    \item Bach wrote in many differing styles to suit differing situations and patrons.
        Therefore the more that you can learn about these different styles (Italian, French, and so on) the better.
\end{enumerate}

\subsubsection{Advice for competitions and examinations}

When you perform a work by Bach in an examination or in a competition, you will often come up against the closed mind of the examiner or adjudicator.
This person may be very critical of your hard won performance.
If they know their business, even though they may not agree with your conclusions, good judges will not penalise you.
Unfortunately this is often not the case, and you will lose marks.
Therefore, avoid playing Bach in these situations.

\subsection{Part 5: Mozart and Haydn}

\subsubsection{Introduction}

I have bracketed Mozart and Haydn together not because of the similarities between them, but because of the profoundly different challenges each presents the performer.

\subsubsection{Mozart}

A famous person once said that ``Mozart is too easy for children and too difficult for adults''.
There is some truth in this statement.
It is my view that there are in fact very few really good Mozarteans, certainly there are less great performers of his music than there are of Bach, Beethoven, Chopin, Brahms etc.

Why should this be?
His music does not present great technical challenges to the able pianist.
The structure of his music is transparent and the musical language he employs is comfortably familiar.

The difficulty lies in our perception of what his music means.
We carry the burden of a tragic figure who died prematurely.
This image haunts the performer, as he/she seeks to find the greatness within the score.
Children have no such problems and happily ``go with the flow''.
Thus, it is my opinion, that the mature players should first rid themselves of their anxious reverance and play the music and not the person.
The way to play his music lies in the notes.

Mozart was perhaps the greatest of all musical craftsmen.
Virtually everything that he wrote flows seamlessly forward, not a note too many or a note too few.
Every phrase in perfect accord within its context.
Every note a melodic gem.
For this reason I feel that the current fashion to decorate his music with ornaments at every available turn is often harmful to the shape of the music and frequently downright tasteless.
It detracts from the purity of the melodic line, melody being the crucial first principal of his music.

Mozart was first and foremost a melodist.
If you treat every single note as part of a melodic line, you will avoid the machine-gun like delivery that so often mars the performing of the repetitive accompanying figuration that is typical of his style.

The melodic lines of Mozart are cunning in the extreme.
Do not allow yourself to be hypnotised into rattling them off as a series of even sparkling scales.
To do this is to miss the point.

The critical point of Mozart's music, the fundamental element that lifts him above his contemporaries and elevates his music to the very highest level is: Mozart's melodies are forever trying to escape from the gravitational pull of the underlying harmony.

They twist and turn, jump and dive, always trying to free themselves from the chains of conventional harmonic progression.
It is this that gives his music that particular yearning quality.
It is this that gives his music its incredible forward momentum.
The harmony chases the melody trying to capture it, as indeed it does at the cadence points, only to see the melody surge ahead again and the chase to be continued into the next section of the work!

Performers: this music really goes.
Avoid gravitas.
Taste the freedom and the beauty of melodic flight.
Save your angst for Brahms.
Be a child.
Chase butterflies - but remember your net has a hole.
No sooner do you have it when it is gone, elusive until the end.
But note this: Children too have their dark and tragic moments.

The real Mozarteans all do this.
They avoid the trap of over-interpretation.
They join with Mozart's free spirit and go where he takes them.
Never bland.
Never mechanistic.
Never reverential.

Mozart was an impresario.
He knew his public.
He knew how to entertain.

As for the Operas: that is another story.
In those psychodramas Mozart explored the human soul as no other composer ever has.
Only occasionally does his piano music enter this tortuous and subtle world.

Remember, we are pianists.
We play piano music.
When Mozart wrote for the piano, he did just that.
Don't look for more than is there.
What is there is more than enough!

\subsubsection{Haydn}

Poor Haydn, forever the poor relation!

Haydn has an image problem!
Always condemned to exist in the shadow of Mozart and Beethoven.
All wrong!

Haydn is the true progenitor of Schubert.

Haydn was the classical composer par-excellence.
I use the word composer in a very particular sense.
His music is composed, constructed, built!
Understand this and you are half way there.

Take any piece by Haydn.
What do you \textbf{see}?
Look carefully.
It will be a marvellous example of the composers art.
The material is laid out before you, clearly and without fuss.
He then proceeds to take it through the twin process of creative examination and exploitation.
One element at a time.
First this way, then that, then into this key, then see what it does in that key: you weren't expecting that were you?
And so it goes on.
The perfect length, striking a perfect balance between the predictable and the unpredictable.
The man was a genius!
His material is always under perfect control.
His treatment of it always appropriate.
Nothing out of place.

Bland?
Boring?
\textbf{Never!}

If doing everything brilliantly is a problem, if appreciating true craftsmanship is a problem, then you do not understand the art of music.
A composer does not have to have a ``story'' to be great, all he needs is supreme talent and skill.
This is Haydn's place is the firmament of greatness.

Roll up your sleeves.
Can you make everything that he has crafted come to life?
Can you strike the right tempo balance between allowing the detail to be revealed and the form to hang together?
In short, can \emph{you}, the performer, match up to Haydn, the composer?!

\subsection{Part 6: Beethoven}

\subsubsection{Introduction}

There has been so much said about Beethoven's music that my contribution may be seen as superfluous.
I feel, however, that what I have to say on the subject could be useful.
So much written about him that it can sometimes be difficult to see ``the wood for the trees''.
I hope that this article contributes some common sense clarity to the subject.

\subsubsection{Some general observations}

Beethoven's works for the piano are generally acknowledged to represent the Everest of the pianist's repertoire.
And so they do.
The range of expression that his piano music covers is far greater than that of any other composer.
The demands he places upon the pianist, both technical and musical are, without doubt, formidable.
This is just as it should be.
The greatest composer of all time (or is it J.S.Bach?) is bound to take even the finest musicians and pianists to their limits.
How then can anyone cope with the responsibility?
Simple!
Be...

\begin{enumerate}
    \item ...Realistic.
        Your contribution is that of an individual.
        No matter how gifted you are, there is simply no way that you are capable of doing complete justice to the work that you are performing/studying.
        His music is just too good.
        All you can do is your best.
        All you can do is ``get close''.
    \item Enjoy the experience.
        Just because this is Beethoven that you are playing doesn't automatically mean that you must go into hair shirt mode; weighted down by the responsibility of your task - as if carrying The Cross of Western Music on your frail and unworthy shoulders.
        This is, still, just music.
        The world is a miserable enough place already without you suffering over a piece, no matter how great.
        This may seem a little ``flip'', but over reverence and over analysis can lead to playing which is, frankly, constipated.
        You don't need me to tell you that there are an awful lot of constipated Beethoven pianists out there, just use your ears.
    \item Shoot from the hip.
        Beethoven was a man who ``went for it''.
        He was a man without guile.
        He was wholehearted, sincere.
        He was a fully functioning, paid up, member of humanity.
        Complete, in his emotional ability to feel what we human beings are all capable of feeling.
        A man utterly incapable of pretense and pretention.
        The world of polite chicanery; the world of social manouvering was alien to this most honest of men.
        Yes, he was subtle!
        Yes, he was emotionally  sophisticated!
        Yes, he was crafty!
        Yes, he was passionate!
        \textbf{But} he was \textbf{not} a ``bullshitter''!
\end{enumerate}

Play what is written, be emotionally honest, direct, and, in the idealistic sense, keep things simple.

\subsubsection{General characteristics}

The most valuable advice I can give to anyone who wishes to play Beethoven's music well, is to make this one observation which is fundamental to the understanding of his work: Beethoven was the first major composer in whose music the performance directions are a fundamental part of the musical structure.

Although many of the markings in his music are open to interpretation and dispute, you should at all times work on the basis that the whole score is prescribed.
Slurs, dynamics, tempi etc., etc.
If he says play fast, he means play fast!

We know through his sketches that he was a meticulous, fanatical worker and it is inconceivable that he left anything to chance.

\begin{itemize}
    \item In the Sonatas, Beethoven covers the whole range of expressive possibilities.
        Crucially, this is not done merely by contrasting material, but by letting/making the material evolve.
        Development is not simply a matter of compositional devises and techniques for Beethoven.
        His development treats the material organically.
        The material itself - first subject, second subject  and so on - almost always, is of itself evolutionary, changing.
        Transforming(!) itself, before our eyes and ears long before the development section has officially arrived.

        It is this characteristic that led in time to what I call his ``monocoque forms''; the completely seamless formal structures that characterise his later works.

        This integrated character that typifies his output was the very stuff of musical revolution.
        Indeed, Beethoven was the most revolutionary of all composers, and did more than any other composer to change the very nature of music, (closely followed by Liszt - see my later articles,).

    \item Beethoven composed with flair and panache.
        Frequently is material is instantly catchy.
        Look for ``the hooks''.
        Look also for the flamboyant touches, he loved sounding good.
        There is no doubt at all, in my mind, that a reincarnated Beethoven would have enthusiastically embraced electronic synthesis and the new technology of the recording studio.

    \item Although Beethoven has given pianists their best music; his best was reserved for the string quartet.

        This observation leads us to another key element which characterises his music: counterpoint!
        The increasingly perfect synthesis of harmonic and contrapuntal techniques is plain to see as you follow his compositional development.
        The juxtaposition of Fugue with melody topped harmonic material in Opus 110. is just an obvious ``in your face'' example.
        This technique, though not always explicit, is an ever present element in his later works.

        Look how many of his pieces are in four parts.
        All his work studying/doing Bach Chorales was in a good cause!

        Then, look at his increasing use of variations.
        He was able to absorb all the best that the past had left into his own language by the time he was ready to write his late, great masterpieces.

    \item Do not under value his early music.
        Here you will find the start of a trail that contains/traces the very same musical material that was to provide the basis for his last works.
        He was, literally, practising for future pieces all the time.

        As you progress, chronologically, through his ouvre, you can grow alongside him in his quest for expressive perfection!

    \item Beethoven was a king of rhythm.
        His ability to write and develop the rhythmic structures that are integral to all his music, on both a micro and macro level, makes him the true precursor of Stravinsky and Boulez.
\end{itemize}

\subsubsection{Final thoughts}

I am not a great fan of the ``Three Period'' notion.
No one is like that!
It is much too arbitrary to limit him to this simplistic invention of some historians.
I am sure that this has led to many anaemic, mechanistically classical performances of his early music.
There is simply no way that his first sonata (Opus 2, No. 1) is Mozartean or Haydnean (sic).

I do not think he was a tragic figure.
He was just isolated by his greatness and, quite simply, could not be bothered with those whom he saw as fools!

\subsubsection{Further reading}

\begin{itemize}
    \item Beethoven - His Spiritual Development by J.W.N,Sullivan
\end{itemize}

\subsubsection{Best edition}

\begin{itemize}
    \item G. Henle Verlag
    \item Agosti's is also well worth having
    \item Schnabel's is interesting, but is so infused with S's own interpretations that you can't see through to the music.
    \item Associated Board is O.K.-ish.
\end{itemize}

\subsubsection{Best recordings}

\begin{itemize}
    \item Freidrich Gulda for everything
    \item Emil Gilels for The Waldstein
    \item Claudio Arrau for Opus 101
    \item Gilels and Radu Lupu for the Concertos
    \item Hungarian String Quartet for the Quartets
\end{itemize}

\end{document}
